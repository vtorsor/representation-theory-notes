\documentclass[12pt]{article}

\usepackage{amssymb,amsmath,amsthm}

% theorem styles (I like everything to have the same counter)
\theoremstyle{plain}
\newtheorem{theorem}{Theorem}[section]
\newtheorem{lemma}[theorem]{Lemma}
\newtheorem{proposition}[theorem]{Proposition}
\newtheorem{corollary}[theorem]{Corollary}
\newtheorem{conjecture}[theorem]{Conjecture}
\newtheorem{exercise}[theorem]{Exercise}
\theoremstyle{definition}
\newtheorem{definition}[theorem]{Definition}
\newtheorem{example}[theorem]{Example}
\theoremstyle{remark}
\newtheorem{remark}[theorem]{Remark}

% some numbering settings
\numberwithin{equation}{section}

\usepackage{fancyhdr}
\usepackage{lastpage}
\setlength{\headheight}{15.2pt}
\renewcommand{\footrulewidth}{0.4pt}% default is 0pt
\setlength{\footskip}{30pt}
\pagestyle{fancy}

\makeatletter
\let\ps@plain\ps@fancy 
\makeatother

\lhead{MATH 742}
\chead{$S_n$ and $\operatorname{GL}_n$}
\rhead{Spring 2023}
\lfoot{Last Revised: \today}
\cfoot{}
\rfoot{\thepage\ of \pageref{LastPage} }

\begin{document}

%%%%%%%%%%%%%%%%%%
%%%%%%%%%%%%%%%%%%
\title{Modules and Wedderburn Theory}
\author{Alexander Duncan}

\maketitle

Modern representation theory is usually phrased in the language of
non-commutative rings/algebras and their modules.
Until now I've avoided this language so I could focus on the basics.
However, the module-theoretic perspective is much more versatile --- even
for finite groups over $\mathbb{C}$.
It is much more urgently needed to discuss representation
theory over other fields. 

The main object is the \emph{group ring} $\mathbb{Z}G$
or \emph{group algebra} $kG$.  We will see that most of the language of
representation theory of finite groups can be efficiently expressed
as special cases of module theory over $kG$.
Then we will use the language to go even further.

Since module theory is part of the standard algebra qual sequence,
I will frequently omit proofs and even many definitions under the
assumption that you should have seen them before.
However, the emphasis in qual courses is usually on
\emph{commutative} rings so some reminders are in order.
I recommend finding a graduate algebra textbook with a good treatment of
modules, such as \cite[\S{10}]{DF} or \cite[\S{III}]{Lang},
to refer to if something is unfamiliar.

The ``basics'' of module-theoretic representation theory is in
almost every text I've referenced, often at a very early stage; 
see \cite{DF}, \cite{Etingof}, \cite{FultonHarris}, \cite{Lang} etc.
Serre's book \cite{Serre}, does not use the module-theoretic
perspective at first, but then abruptly changes gears in chapter 6 and
assumes you've seen Wedderburn theory (developed below).
I will also draw heavily from \cite[\S{12,13}]{AlperinBell},
\cite{AlperinLRT}, and \cite{CurtisReiner}.

%%%%%%%%%%%%%%%%%%
%%%%%%%%%%%%%%%%%%

\section{The Group Ring}

Throughout, we assume that all rings and algebras are unital (have an identity
element) but are \emph{not necessarily} commutative.

\begin{definition}
Suppose $G$ is a finite group.
The \emph{group ring} $\mathbb{Z}G$ is the free abelian group
on $G$ with multiplication given by
\[
\left( \sum_{g \in G} a_g g \right) \cdot
\left( \sum_{g \in G} b_g g \right)
:= \sum_{g \in G} \sum_{h \in G} a_gb_h gh
\]
for integers $\{a_g\}_{g\in G}$ and $\{b_g\}_{g\in G}$.
More generally, given a commutative ring $k$,
the \emph{group algebra} $kG$ is the free $k$-module on $G$
with multiplication as above.
\end{definition}

One checks immediately that $\mathbb{Z}G$ is a non-commutative ring with
identity $1$ corresponding to the basis element of the identity in $G$.
Similarly, $kG$ is a $k$-algebra with identity $1$.
Since $1 \in G$ corresponds to $1 \in \mathbb{Z}G$,
the identification of the basis of $\mathbb{Z}G$ with the elements of
$G$ is mostly harmless if $G$ is written multiplicatively.

\begin{example}
Suppose $G=D_6=\langle s,r \mid s^2, r^3, (sr)^2 \rangle$
is the dihedral group of order $6$.
We have the following computation in $\mathbb{Z}G$:
\begin{align*}
&(3 + 2s + 5r)(r-sr + 5r^2)\\
=& 3r -3sr +15r^2 + 2sr -2r+10sr^2 +5r^2 -5s+25\\
=& 25 -5s + r + 20r^2 -sr +10sr^2
\end{align*}
\end{example}

\begin{example}
Let $G = \langle r \mid r^n \rangle$ be a cyclic group of order $n$.
Thus $G$ has basis $\{1,r,r^2,\ldots, r^{n-1} \}$
where $r^i\cdot r^j=r^{i+j}$ subject to the relation $r^n=1$.
In other words, $\mathbb{Z}G$ is isomorphic to the ring
$\mathbb{Z}[x]/(x^n-1)$.
\end{example}

Note that using additive notation $\mathbb{Z}/n\mathbb{Z}$
for the cyclic group of order $n$ would be dangerously
ambiguous in the example above.
For this reason, we usually use ``exponential notation''
for group rings of additive groups.
If $a+b=c$ in an additive group, then
we write $x^ax^b=x^{a+b}=x^c$ in the group ring.

\begin{exercise}
Prove that $\mathbb{Z}G$ is commutative if and only $G$ is abelian.
\end{exercise}

It is useful to write the coefficients of the product as a formula of
the multiplicands.  Specifically, if
\[
\left( \sum_{g \in G} a_g g \right) \cdot
\left( \sum_{g \in G} b_g g \right)
= \sum_{g \in G} c_g g
\]
then
\[
c_g = \sum_{h \in G} a_hb_{h^{-1}g}
\]
for every $g \in G$.

Another description of $\mathbb{Z}G$ is as the set
$\operatorname{Hom}_{\mathrm{Set}}(G,\mathbb{Z})$
of functions from $G$ to $\mathbb{Z}$.
However, the ring structure is \emph{not} the ``obvious one.''
The additive structure is indeed pointwise addition,
but we use the \emph{convolution product}
of two functions $f_1,f_2$ defined via
\[
(f_1 \ast f_2)(g) = \sum_{ij=g} f_1(i)f_2(j)
\]
for $g \in G$ and the indices in the sum $i,j$
vary over all pairs of elements of $G$ such that $ij=g$.
The isomorphism with $\mathbb{Z}G$ is by identifying each basis element
$g \in \mathbb{Z}G$ with the characteristic function $\chi_g$
in $\operatorname{Hom}_{\mathrm{Set}}(G,\mathbb{Z})$
satisfying $\chi_g(h) = \delta_{gh}$ for all $h \in G$.

\bibliographystyle{alpha}
\bibliography{rep_theory}

\end{document}


