\documentclass[12pt]{article}

\usepackage{amssymb,amsmath,amsthm}

% theorem styles (I like everything to have the same counter)
\theoremstyle{plain}
\newtheorem{theorem}{Theorem}[section]
\newtheorem{lemma}[theorem]{Lemma}
\newtheorem{proposition}[theorem]{Proposition}
\newtheorem{corollary}[theorem]{Corollary}
\newtheorem{conjecture}[theorem]{Conjecture}
\newtheorem{exercise}[theorem]{Exercise}
\theoremstyle{definition}
\newtheorem{definition}[theorem]{Definition}
\newtheorem{example}[theorem]{Example}
\theoremstyle{remark}
\newtheorem{remark}[theorem]{Remark}

% some numbering settings
\numberwithin{equation}{section}

\usepackage{fancyhdr}
\usepackage{lastpage}
\setlength{\headheight}{15.2pt}
\renewcommand{\footrulewidth}{0.4pt}% default is 0pt
\setlength{\footskip}{30pt}
\pagestyle{fancy}

\makeatletter
\let\ps@plain\ps@fancy 
\makeatother

\lhead{MATH 742}
\chead{Modules and Wedderburn Theory}
\rhead{Spring 2023}
\lfoot{Last Revised: \today}
\cfoot{}
\rfoot{\thepage\ of \pageref{LastPage} }

\begin{document}

%%%%%%%%%%%%%%%%%%
%%%%%%%%%%%%%%%%%%
\title{Modules and Wedderburn Theory}
\author{Alexander Duncan}

\maketitle

Modern representation theory is usually phrased in the language of
non-commutative rings/algebras and their modules.
Until now I've avoided this language so I could focus on the basics.
However, the module-theoretic perspective is much more versatile --- even
for finite groups over $\mathbb{C}$.
It is much more urgently needed to discuss representation
theory over other fields. 

The main object is the \emph{group ring} $\mathbb{Z}G$
or \emph{group algebra} $kG$.  We will see that most of the language of
representation theory of finite groups can be efficiently expressed
as special cases of module theory over $kG$.
Then we will use the language to go even further.

Since module theory is part of the standard algebra qual sequence,
I will frequently omit proofs and even many definitions under the
assumption that you should have seen them before.
However, the emphasis in qual courses is usually on
\emph{commutative} rings so some reminders are in order.
I recommend finding a graduate algebra textbook with a good treatment of
modules, such as \cite[\S{10}]{DF} or \cite[\S{III}]{Lang},
to refer to if something is unfamiliar.

The ``basics'' of module-theoretic representation theory is in
almost every text I've referenced, often at a very early stage; 
see \cite{DF}, \cite{Etingof}, \cite{FultonHarris}, \cite{Lang} etc.
Serre's book \cite{Serre}, does not use the module-theoretic
perspective at first, but then abruptly changes gears in chapter 6 and
assumes you've seen Wedderburn theory (developed below).
I will also draw heavily from \cite[\S{12,13}]{AlperinBell},
\cite{AlperinLRT}, and \cite{CurtisReiner}.

%%%%%%%%%%%%%%%%%%
%%%%%%%%%%%%%%%%%%

\section{The Group Ring}

Throughout, we assume that all rings and algebras are unital (have an identity
element) but are \emph{not necessarily} commutative.
We will reserve $k$ for a commutative ring, which will usually be a
field or $\mathbb{Z}$.

\begin{definition}
Suppose $G$ is a finite group.
The \emph{group ring} $\mathbb{Z}G$ is the free abelian group
on $G$ with multiplication given by
\[
\left( \sum_{g \in G} a_g g \right) \cdot
\left( \sum_{g \in G} b_g g \right)
:= \sum_{g \in G} \sum_{h \in G} a_gb_h gh
\]
for integers $\{a_g\}_{g\in G}$ and $\{b_g\}_{g\in G}$.
More generally, given a commutative ring $k$,
the \emph{group algebra} $kG$ is the free $k$-module on $G$
with multiplication as above.
\end{definition}

One checks immediately that $\mathbb{Z}G$ is a non-commutative ring with
identity $1$ corresponding to the basis element of the identity in $G$.
Similarly, $kG$ is a $k$-algebra with identity $1$.
Since $1 \in G$ corresponds to $1 \in \mathbb{Z}G$,
the identification of the basis of $\mathbb{Z}G$ with the elements of
$G$ is mostly harmless if $G$ is written multiplicatively.

\begin{example}
Suppose $G=D_6=\langle s,r \mid s^2, r^3, (sr)^2 \rangle$
is the dihedral group of order $6$.
We have the following computation in $\mathbb{Z}G$:
\begin{align*}
&(3 + 2s + 5r)(r-sr + 5r^2)\\
=& 3r -3sr +15r^2 + 2sr -2r+10sr^2 +5r^2 -5s+25\\
=& 25 -5s + r + 20r^2 -sr +10sr^2
\end{align*}
\end{example}

\begin{example}
Let $G = \langle r \mid r^n \rangle$ be a cyclic group of order $n$.
Thus $G$ has basis $\{1,r,r^2,\ldots, r^{n-1} \}$
where $r^i\cdot r^j=r^{i+j}$ subject to the relation $r^n=1$.
In other words, $\mathbb{Z}G$ is isomorphic to the ring
$\mathbb{Z}[x]/(x^n-1)$.
\end{example}

Note that using additive notation $\mathbb{Z}/n\mathbb{Z}$
for the cyclic group of order $n$ would be dangerously
ambiguous in the example above.
For this reason, we usually use ``exponential notation''
for group rings of additive groups.
If $a+b=c$ in an additive group, then
we write $x^ax^b=x^{a+b}=x^c$ in the group ring.

\begin{exercise}
Prove that $\mathbb{Z}G$ is commutative if and only $G$ is abelian.
\end{exercise}

It is useful to write the coefficients of the product as a formula of
the multiplicands.  Specifically, if
\[
\left( \sum_{g \in G} a_g g \right) \cdot
\left( \sum_{g \in G} b_g g \right)
= \sum_{g \in G} c_g g
\]
then
\[
c_g = \sum_{h \in G} a_hb_{h^{-1}g}
\]
for every $g \in G$.

Another description of $\mathbb{Z}G$ is as the set
$\operatorname{Hom}_{\mathrm{Set}}(G,\mathbb{Z})$
of functions from $G$ to $\mathbb{Z}$.
However, the ring structure is \emph{not} the ``obvious one.''
The additive structure is indeed pointwise addition,
but we use the \emph{convolution product}
of two functions $f_1,f_2$ defined via
\[
(f_1 \ast f_2)(g) = \sum_{ij=g} f_1(i)f_2(j)
\]
for $g \in G$ and the indices in the sum $i,j$
vary over all pairs of elements of $G$ such that $ij=g$.
The isomorphism with $\mathbb{Z}G$ is by identifying each basis element
$g \in \mathbb{Z}G$ with the characteristic function $\chi_g$
in $\operatorname{Hom}_{\mathrm{Set}}(G,\mathbb{Z})$
satisfying $\chi_g(h) = \delta_{gh}$ for all $h \in G$.

\subsection{Modules of the Group Ring}

The relevance of the group ring to group theory is evident from the
following proposition:

\begin{proposition} \label{prop:equivalence}
Let $G$ be a finite group and $k$ a field.
The structure of a representation of $G$ over the field $k$
is equivalent to the structure of a left $kG$-module.
Finite-dimensional representations correspond to
finitely generated modules.
Moreover, $G$-equivariant linear transformations are
exactly the $kG$-module homomorphisms.
\end{proposition}

\begin{proof}[Proof of Proposition~\ref{prop:equivalence}]
Let $M$ be a left $kG$-module.
In particular, $M$ is a $k$-vector space.
We construct a linear representation $\rho : G \to \operatorname{GL}(M)$
via $\rho_g(m)=gm$ for $g \in G$ and $m\in M$.
Conversely, given a linear representation $\sigma : G \to
\operatorname{GL}(V)$, we endow $V$ with a $kG$-module structure via
\[
\left( \sum_{g \in G} a_g g \right) v := \sum_{g \in G} a_g \sigma_g(v)
\]
for $\{a_g\}$ from $k$ and $v \in V$.
These are mutually inverse operations.

If $M$ is a finitely-generated left $kG$-module,
say by $\{s_1,\ldots,s_n\}$, then $M$ is spanned as a $k$-vector space
by $\{ gs_i \mid g \in G, 1 \le i \le n \}$.
Thus $\rho$ is a finite-dimensional representation.
Conversely, if $V$ is finite-dimensional, then a basis for $V$
as a $k$-vector space is a fortiori a generating set for $V$ as a
$kG$-module.

Finally, we observe that a $G$-equivariant linear transformation
$f : (V,\rho) \to (W,\sigma)$ of representations are exactly the $kG$-module
homomorphisms under the above correspondence.
Indeed,
\begin{align*}
&f\left( \left( \sum_{g\in G} c_g g \right) v \right)\\
=& f\left( \sum_{g\in G} c_g \rho_g(v) \right)\\
=& \sum_{g\in G} c_g f\left( \rho_g(v) \right)\\
=& \sum_{g\in G} c_g \sigma_g\left(f(v)\right)\\
=& \left(\sum_{g\in G} c_g g\right) f(v)
\end{align*}
where we used the fact that $f$ is linear and $G$-equivariant.
\end{proof}

Immediately, we see that subrepresentations correspond to submodules,
quotient representations correspond to quotient modules, and direct sums
correspond to direct sums.
However, while $V \otimes_k W$ and $\operatorname{Hom}_k(V,W)$ have
induced $kG$-module structures, they are \emph{not} the same as the
module-theoretic constructions $V \otimes_{kG} W$
and $\operatorname{Hom}_{kG}(V,W)$ discussed below.
Moreover, the ``trivial module'' is the zero representation,
\emph{not} the ``trivial representation''.

\begin{example}
Every ring is a module under itself.
The group algebra $kG$ viewed as a $kG$-module is just the regular
representation of $G$.
\end{example}

Another important subtlety in the above is that the dimension of a
representation is \emph{not} in general the same as the minimal number
of generators of the corresponding $kG$-module.

\begin{definition}
Let $R$ be a ring and $M$ an $R$-module.
We say $M$ is \emph{simple} if $M\ne 0$ has no submodules except for
$0$ and $M$ itself.
We say $M$ is \emph{indecomposable} if $M$ cannot be written as a direct sum of
non-zero submodules.
We say $M$ is \emph{semisimple} if $M$ is a direct sum of simple
submodules.
\end{definition}

A $kG$-module is indecomposable if it is indecomposable as a
representation.
A $kG$-module is simple if and only if it is irreducible
as a representation of $G$.
In particular, irreducible representations can be
generated by one element as a $kG$-module (though the converse may not
be true).

\begin{theorem}[Maschke's Theorem restated]
If $G$ is finite of order coprime to the characteristic of a field $k$,
then every finitely-generated $kG$-module is semisimple.
\end{theorem}

\subsection{Center of the Group Ring}

The center of a group ring is of special importance.

\begin{definition}
Let $R$ be a ring.
The \emph{center} of $R$ is the subset
\[
Z(R) := \{ z \in R \mid rz=zr \textrm{ for all } r \in R \},
\]
which is a commutative subring of $R$.
\end{definition}

Note that we also have the notion of the center $Z(G)$ of a group $G$.
Indeed, the group ring of the center of a group is contained in the
center of the group ring; in other words
\[
kZ(G) \subseteq Z(kG).
\]
However, equality holds only when $G=Z(G)$, which is a consequence of the
following result.

\begin{proposition}
The center $Z(kG)$ of the group algebra $kG$ has basis
\[
\left\{
\sum_{g \in K} g\ \middle|\ K \in \mathcal{K}
\right\}
\]
where $\mathcal{K}$ is the set of conjugacy classes of a finite group.
\end{proposition}

\begin{proof}
Let $K$ be a conjugacy class in $\mathcal{K}$.
Note that $g \in K$ implies
$hgh^{-1} \in K$.
Thus we observe that
\[
h\left(\sum_{g \in K} g\right)
=\left(\sum_{g \in K} g\right) h
\]
for all $h \in H$. 
Conversely, any element $x$ of the group algebra
satisfying the property $hx=xh$ must have the same coefficient
on basis elements belonging to the same conjugacy class.
\end{proof}

\begin{example}
Suppose $G=D_6=\langle s,r \mid s^2, r^3, (sr)^2 \rangle$
is the dihedral group of order $6$.
We determine that
$Z(\mathbb{Z}G)$ has basis
\[
\{ 1, a = s+sr+sr^2, b = r+r^2 \}.
\]
The multiplication table is determined by a few calculations:
\begin{align*}
a^2 &= 3+3b\\
ab=ba &= 2a\\
b^2 &=2+b.
\end{align*}
Thus, the structure of the center is a bit obscure in this basis.
\end{example}

When $k=\mathbb{C}$ the center is especially easy to describe.
We will see in the next section that $Z(\mathbb{C}G) \cong \mathbb{C}^{|K|}$
as $\mathbb{C}$-algebras.  However, this is not at all obvious in the
basis described above.

\subsection{Decomposition of the Complex Group Algebra}

Let $G$ be a finite group.
Let $W_1,\ldots,W_r$ be the distinct
complex irreducible representations of $G$
and let $n_1,\ldots, n_r$ be their dimensions.

\begin{theorem}
There is a canonical isomorphism
\[
\mathbb{C}G \cong \bigoplus_{i=1}^r \operatorname{End}_{\mathbb{C}}(W_i)
\]
of $\mathbb{C}$-algebras.
\end{theorem}

\begin{proof}
If $V$ is a $kG$-module and $x$ is an element of $kG$,
then the multiplication map $v \mapsto xv$ is an endomorphism
of $V$.
This gives a map from $\mathbb{C}G$ to each
$\operatorname{End}_C(W_i)$ and we obtain
a map to the direct sum.
The map is injective since the action on the regular representation
is faithful.
Since both algebras have dimension $n_1^2+\cdots+n_r^2$,
this must be an isomorphism.
\end{proof}

Note that for any complex vector space $V$ of dimension $n$, we have
a (non-canonical) isomorphism
$\operatorname{End}(V) \cong \operatorname{M}_n(\mathbb{C})$.
Thus, we can rewrite the theorem above as:  

\begin{corollary} \label{cor:fourier_as_matrices}
$\displaystyle
\mathbb{C}G \cong \bigoplus_{i=1}^r \operatorname{M}_{n_i}(\mathbb{C})$
\end{corollary}

Recall that the center of a matrix algebra
$\operatorname{M}_n(\mathbb{C})$ is just the subalgebra of scalar
matrices, which is isomorphic to $\mathbb{C}$.
In view of Corollary~\ref{cor:fourier_as_matrices},
we have
\[
Z(\mathbb{C}G) \cong Z\left( \bigoplus_{i=1}^r \operatorname{M}_{n_i}(\mathbb{C}) \right)
\cong \mathbb{C}^r,
\]
which is an isomorphism of $\mathbb{C}$-algebras.

Thus we have two bases of $Z(\mathbb{C}G)$: one indexed by conjugacy
classes and one indexed by irreducible representations.
The character table of $G$ is exactly the change of basis matrix between
these two bases.

\begin{example}
Let $G = \mathbb{Z}/3\mathbb{Z}$ be the cyclic group of order $3$.
Note $kG \cong k[x]/(x^3-1)$ for any field $k$.
We fine that $\mathbb{R}G \cong \mathbb{R} \oplus \mathbb{C}$,
so the corollary is more subtle over non-closed fields.
Moreover, the element $x-1$ in $\mathbb{F}_3G$ is nilpotent,
so things are potentially much worse when Maschke's theorem does not
hold.
\end{example}

\subsection{Idempotents of the Group Ring}

Earlier in the course, we have seen several examples of projections in
$\operatorname{End}_k^G(V)$ for various representations $(V,\rho)$.
For example, the projection $\pi : V \to V^G$ onto the invariant
subspace has the formula
\[
\pi(v) = \frac{1}{|G|} \sum_{g \in G} \rho_g(v)
\]
for $v \in V$.
Other examples include the Young projectors, Young symmetrizers,
and the projections onto isotypic components.
These are all more naturally considered as coming from elements
of the group algebra.

Given a representation $\rho : G \to \operatorname{GL}(V)$
and an element
\[
f = \sum_{g \in G} c_g g
\]
in the group algebra $kG$,
define $\widehat{f}(\rho) \in \operatorname{End}(V)$ via
\[
\widehat{f}(\rho)(v) = \sum_{g \in G} c_g \rho_g(v)
\]
for all $v \in V$.
Thus, the projection $\pi$ above is simply the endomorphism
$\widehat{f}(\rho)$
where $f = |G|^{-1} \sum_{g \in G} g$.
In practice, we will simply write $f$ instead of $\widehat{f}(\rho)$
since there is rarely danger of confusion.

\begin{definition}
Let $R$ be a ring.  An element $e$ is an \emph{idempotent} if $e^2=e$.
Two idempotents $e_1,e_2$ are \emph{orthogonal} if $e_1e_2=e_2e_1=0$.
An idempotent $e$ is \emph{central} if $e$ is contained in the center
$Z(R)$ of $R$.
A \emph{primitive idempotent} is a non-zero idempotent
that is not a sum of two non-zero orthogonal idempotents.
A \emph{primitive central idempotent} is a non-zero central idempotent
that is not a sum of two non-zero orthogonal central idempotents.
\end{definition}

Warning: a primitive central idempotent is often not a central primitive
idempotent!  Sometimes the term ``centrally primitive'' is used
instead of ``primitive central,'' which is either more or less confusing
depending on which side you rolled out of the bed in the morning.

\begin{example}
Consider the matrix algebra $\operatorname{M}_2(\mathbb{C})$
and set
\[
a = \begin{pmatrix} 1 & 0 \\ 0 & 0 \end{pmatrix},
b = \begin{pmatrix} 0 & 0 \\ 0 & 1 \end{pmatrix},
c = \begin{pmatrix} \frac{1}{2} & \frac{1}{2} \\ \frac{1}{2} & \frac{1}{2} \end{pmatrix},
\textrm{ and }
d = \begin{pmatrix} \frac{1}{2} & -\frac{1}{2} \\ -\frac{1}{2} &
\frac{1}{2} \end{pmatrix}.
\]
Each of $a,b,c,d$ are primitive idempotents, but none are central.
We see that $a,b$ are orthogonal and $c,d$ are orthogonal.
The identity matrix $1$ is the unique non-zero central idempotent,
but it is not a primitive idempotent since $a+b=1$ or $c+d=1$.
However, the identity $1$ is a primitive central idempotent.
\end{example}

We will see later that idempotents are a useful module theoretic
language for understanding decompositions of representations.
First, observe that if $p \in kG$ is an idempotent,
then $\widehat{p}(\rho)$ is a projection for any representation $\rho$.
(The converse is not true, consider the direct sum of two copies the regular
representation and the projection onto a summand.)
Second, if $p$ is a primitive idempotent, then the corresponding
projection $\pi$ on the regular representation has image an
indecomposable subrepresentation.

For the complex group ring, we can even say more.
From the isomorphism
\[
\mathbb{C}G \cong \bigoplus_{i=1}^r \operatorname{End}(W_i)
\]
we see that the central idempotents of $\mathbb{C}G$
are the possible sums of the identities of each
$\operatorname{End}(W_i)$.
In particular, the primitive central idempotents correspond
to the projections on the isotypic components.

\section{Bimodules and Tensor Products}

When $R$ is a commutative ring, we will often just say $R$-module without
specifying whether it is left or right.

\begin{definition}
If $k$ is a \emph{commutative} ring, then we define an
\emph{$k$-algebra} as a (not-necessarily commutative) ring $R$ along
with a ring homomorphism $\pi : k \to R$ such that $\pi(k) \subseteq Z(R)$.
An \emph{$k$-algebra homomorphism} $f : R \to S$ is simply a ring
homomorphism such that $\pi_S = f \circ \pi_R$.
\end{definition}

Equivalently, an $k$-algebra is both a $k$-module and a ring such
that the structures are compatible.
Observe that this alternative characterization breaks down if $k$ is not
in the center of the overring $R$, since it is not clear whether
$R$ should be a left or right $R$-module.
Thus, the restriction to commutative base rings in the center of the
overring is fairly reasonable.

Over non-commutative rings, there are some subtleties to homomorphisms
and tensor products that can be ignored in the commutative setting.
Here we discuss some of these subtleties.

\begin{definition}
Given a ring $R$, the \emph{opposite ring} $R^{\mathrm{op}}$ has
the same underlying abelian group, but the multiplication is in the
reverse order; in other words, $a \cdot_{\mathrm{op}} b := ba$.
\end{definition}

Note that commutative rings are canonically isomorphic to their
opposite rings by the identity map.
Since the base ring $k$ of a $k$-algebra is always commutative,
the opposite ring of a $k$-algebra is still a $k$-algebra.

The transpose map $A \mapsto A^T$ gives a canonical isomorphism
of the matrix ring $\operatorname{M}_n(k)$ with its opposite
$\operatorname{M}_n(k)^{\mathrm{op}}$.
The transpose map $g \to g^{-1}$ gives a canonical isomorphism
of the group ring $\mathbb{Z}G$ with its opposite
$\mathbb{Z}G^{\mathrm{op}}$.

An important example for our purposes shows that we cannot expect
canonical isomorphisms in general:

\begin{example}
Let $V$ be a finite-dimensional vector space with dual $V^\vee$
and consider the ring $\operatorname{End}(V)$.
We have a canonical isomorphism
\[
\psi : \operatorname{End}(V)^{\mathrm{op}} \to \operatorname{End}(V^\vee)
\]
via $\psi(f)(g)=g \circ f$ for $f \in
\operatorname{End}(V)^{\mathrm{op}}$ and $g \in V^\vee$.
Moreover, we obtain a (non-canonical) isomorphism between
$\operatorname{End}(V)^{\mathrm{op}}$ and $\operatorname{End}(V)$
by way of a choice of isomorphism $V \cong V^\vee$.
\end{example}

We will see later that even \emph{division} rings are not necessarily
isomorphic to their opposite.  Here is a small example for the
impatient: 

\begin{exercise}
Consider the subring $R \subseteq \operatorname{M}_2(\mathbb{Q})$
given by
\[
R = \left\{ \begin{pmatrix} a & b\\ 0 & c \end{pmatrix}\ \middle|\
a \in \mathbb{Z}, b \in \mathbb{Q}, c \in \mathbb{Q} \right\}.
\]
Prove that $R$ is not isomorphic to $R^{\mathrm{op}}$.
\end{exercise}

Observe that every left $R$-module is a right $R^{\mathrm{op}}$-module
via the scalar multiplication $m \cdot_{\mathrm{op}} r := rm$.
Similarly, every right $R$-module is a left $R^{\mathrm{op}}$-module.
In particular, the distinction between left and right
$R$-modules is inconsequential precisely when $R$ is commutative.

\begin{definition}
Let $R$ and $S$ be rings.  An \emph{$(R,S)$-bimodule} $M$ is a left
$R$-module that is also a right $S$-module such that $(rm)s=r(ms)$ for
all $r$.
\end{definition}

\begin{exercise}
Show that being an $(R,S)$-bimodule is equivalent to being a
left $R \times S^{\mathrm{op}}$-module, and also to being
a right $R^{\mathrm{op}} \times S$-module.
\end{exercise}

Bimodules are a quite natural object:

\begin{example}
The ring $R$ is in particular an $(R,R)$-bimodule.
\end{example}

\begin{example}
Let $V$ and $W$ be $k$-modules.  The set of homomorphisms
$\operatorname{Hom}_k(V,W)$ is a
$(\operatorname{End}_k(W),\operatorname{End}_k(V))$-bimodule
via $fgh := f \circ g \circ h$.
\end{example}

Every left $R$-module has a canonical structure of a
$(R,\mathbb{Z})$-bimodule; indeed even a $(R,Z(R))$-bimodule where
$Z(R)$ is the center of $R$.
Similarly, every right $R$-module has a canonical structure of
a $(Z(R),R)$-bimodule.
As a consequence, if $R$ is a $k$-algebra, then every left or right
$R$-module is canonically both a left and right $k$-module.

\begin{definition}
If $M$ and $N$ are left $R$-modules, then
\[
\operatorname{Hom}_R(M,N) = \operatorname{Hom}_{R-\operatorname{mod}}(M,N)
\]
is the set of left $R$-module homomorphisms $f : M \to N$.
If $M$ and $N$ are right $R$-modules, then
\[
\operatorname{Hom}_{\operatorname{mod}-R}(M,N)
\]
is the set of right $R$-module homomorphisms $f : M \to N$.
\end{definition}

Suppose that $S$ and $T$ are rings,
$M$ is an $(R,S)$-bimodule, and $N$ is an $(R,T)$-bimodule.
Then $\operatorname{Hom}_R(M,N)$ has a $(S,T)$-bimodule structure
via $(s\phi t)(m):=\phi(ms)t$ for $s \in S, m \in M, t\in T$.
Similarly, if $M$ is an $(S,R)$-bimodule and $N$ is a $(T,R)$-bimodule,
then $\operatorname{Hom}_{\operatorname{mod}-R}(M,N)$
has a $(T,S)$-bimodule structure.

Thus, when $R$ is commutative, $\operatorname{Hom}_R(M,N)$
has a canonical $R$-module structure due to the canonical
$(R,R)$-bimodule structure on $M$ (and $N$).
More generally, $\operatorname{Hom}_R(M,N)$ has only a $Z(R)$-module
structure.
If $R$ is a $k$-algebra, then $\operatorname{Hom}_R(M,N)$
is canonically a $k$-module.
At the very least $\operatorname{Hom}_R(M,N)$ is always an
abelian group due to the canonical $\mathbb{Z}$-module structure.

\begin{example}
Suppose $V$ and $W$ are representations of a finite group $G$
over a field $k$.
Then
\[
\operatorname{Hom}_{kG}(V,W) = \operatorname{Hom}^G_k(V,W)
\]
is the $k$-vector space of $G$-equivariant linear transformations
$f : V \to W$.
\end{example}

\begin{definition}
Suppose $M$ is right $R$-module, $N$ is a left $R$-module,
and $A$ is an abelian group.
A group homomorphism $f: M \times N \to A$ is \emph{$R$-balanced}
if $f(mr,n)=f(m,rn)$ for all $r \in R$, $m \in M$, and $n \in N$.
A \emph{tensor product} $M \otimes_R N$ is an abelian group
together with a $R$-balanced group homomorphism
$\pi : M \times N \to M \otimes_R N$ such that
for any $R$-balanced group homomorphism
$\phi : M \times N \to A$ there exists a unique
group homomorphism $\psi :  M \otimes_R N \to A$
such that $\phi = \psi \circ \pi$.
\end{definition}

Once again, we have:

\begin{proposition}
Tensor products exist and are unique up to unique isomorphism.
\end{proposition}

Just like with hom-sets, the tensor product carries additional
structures when the ingredients are bimodules.
If $M$ is an $(S,R)$-bimodule and $N$ is an $(R,U)$-bimodule,
then $M \otimes_R N$ is an $(S,U)$-bimodule via the multiplication
$s(m\otimes n)r := (sm) \otimes (nr)$ extended by linearity.
In particular, if $R$ is commutative, and $M$ and $N$ are both
\emph{left} $R$-modules, then $M \otimes_R N$ still makes sense.
Also, if $R$ is a $k$-algebra, then $M \otimes_R N$
has a canonical $k$-module structure.

We are now in a position to state an important result:

\begin{theorem}[Tensor-Hom Adjunction]
Let $R$, $S$, $U$, $V$ be rings.
Let $M$ be an $(R,S)$-bimodule, $N$ be an $(S,U)$-bimodule,
and $P$ be an $(R,V)$-bimodule.
Then there is a natural isomorphism
\[
\operatorname{Hom}_R(M \otimes_S N,P)
\cong \operatorname{Hom}_S(N,\operatorname{Hom}_R(M,P))
\]
 of $(U,V)$-bimodules.
\end{theorem}

Rarely are all bimodule structures needed at once!
The most important special case is when $M$ is an $(R,S)$-bimodule,
$N$ is a left $S$-module and $P$ is left $R$-module;
the resulting isomorphism is then just of abelian groups.

One of the main applications of the Tensor-Hom Adjunction is
understanding restriction and extension of scalars.

\begin{definition}
Let $f : R \to S$ be a ring homomorphism.
If $N$ is a left $S$-module, then the \emph{restriction of scalars
of $N$} is the left $R$-module structure on $N$ given by
$r \cdot_R n := f(r)n$ for $r \in R$ and $n \in N$.
We often denote the restriction of scalars by ${}_RN$
or${}_fN$ to emphasize the distinction with $N$ as a left $S$-module.
\end{definition} 

When $f$ is the inclusion of a subring, one can think of the restriction
of scalars as ``forgetting'' some of the structure of $N$.
For example, a complex vector space $\mathbb{C}^n$ becomes the real
vector space $\mathbb{R}^{2n}$ under the inclusion $\mathbb{R} \to
\mathbb{C}$.

\begin{example}
If $H$ is a subgroup of a group $G$, the we have an inclusion
$kH \to kG$ of group algebras.
Let $V$ be a representation of a finite group $G$ over a field $k$.
Then $V$ has the structure of a left $kG$-module.
The restriction of scalars ${}_{kH}V$ is the left $kH$-module
corresponding to $\operatorname{Res}_H^G V$. 
\end{example}

Somewhat trickier is going in the other direction.

\begin{definition}
Let $f : R \to S$ be a ring homomorphism and $M$ be an $R$-module.
Let $S_R$ be the corresponding $(S,R)$-bimodule structure on $S$
and let ${}_RS$ be the corresponding $(R,S)$-bimodule structure on $S$.
The \emph{extension of scalars of $M$}
is the left $S$-module $S_R \otimes_R M$.
The \emph{coextension of scalars of $M$}
is the left $S$-module $\operatorname{Hom}_R({}_RS,M)$.
\end{definition}

The Tensor-Hom adjunction shows that extension of scalars
and restriction of scalars are adjoint.
Indeed, viewing $S$ as an $(S,R)$-bimodule,
we have $\operatorname{Hom}_S(S,M) \cong {}_RM$.
Thus, the adjunction becomes
\[
\operatorname{Hom}_R(S \otimes_R N,M)
\cong \operatorname{Hom}_R(N,{}_RM)
\]
for a left $R$-module $N$ and a left $S$-module $M$.

Once again, these have immediate connections to representation theory:

\begin{example}
If $H$ is a subgroup of a group $G$, the we have an inclusion
$kH \to kG$ of group algebras.
Let $W$ be a representation of a finite group $H$ over a field $k$.
Then $W$ has the structure of a left $kH$-module.
The extension of scalars $kG \otimes_{kH} W$
is the left $kG$-module corresponding to $\operatorname{Ind}_H^G W$.
\end{example} 

To see why extension of scalars corresponds to induction, we just need
to recall that we essentially \emph{defined} induction to be the linear
representation that satisfied (a version of) Frobenius reciprocity.
The Tensor-Hom adjunction gives us an isomorphism of $k$-vector spaces
\[
\operatorname{Hom}_{kG}(kG \otimes_{kH} W,V)
\cong \operatorname{Hom}_{kH}(W,{}_{kH}V))
\]
which is exactly the isomorphism
\[
\operatorname{Hom}_k^G(\operatorname{Ind}_H^G W,V)
\cong \operatorname{Hom}_k^H(W,\operatorname{Res}_H^G V))
\]
from Frobenius Reciprocity!

We leave it as an exercise to check that coextension of scalars
corresponds to coinduction.  Indeed, the distinction is not important
(at least in our setting) in view of the following:

\begin{exercise}
Suppose $H$ is subgroup of finite group $G$ and $W$ is a $kH$-module.
Prove that
\[
kG \otimes_{kH} W \cong \operatorname{Hom}_{kH}(kG,W)
\]
as left $kG$-modules.
\end{exercise}

\bibliographystyle{alpha}
\bibliography{rep_theory}

\end{document}


