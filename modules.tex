\documentclass[12pt]{article}

\usepackage{amssymb,amsmath,amsthm}

% theorem styles (I like everything to have the same counter)
\theoremstyle{plain}
\newtheorem{theorem}{Theorem}[section]
\newtheorem{lemma}[theorem]{Lemma}
\newtheorem{proposition}[theorem]{Proposition}
\newtheorem{corollary}[theorem]{Corollary}
\newtheorem{conjecture}[theorem]{Conjecture}
\newtheorem{exercise}[theorem]{Exercise}
\theoremstyle{definition}
\newtheorem{definition}[theorem]{Definition}
\newtheorem{example}[theorem]{Example}
\theoremstyle{remark}
\newtheorem{remark}[theorem]{Remark}

% some numbering settings
\numberwithin{equation}{section}

\usepackage{fancyhdr}
\usepackage{lastpage}
\setlength{\headheight}{15.2pt}
\renewcommand{\footrulewidth}{0.4pt}% default is 0pt
\setlength{\footskip}{30pt}
\pagestyle{fancy}

\makeatletter
\let\ps@plain\ps@fancy 
\makeatother

\lhead{MATH 742}
\chead{$S_n$ and $\operatorname{GL}_n$}
\rhead{Spring 2023}
\lfoot{Last Revised: \today}
\cfoot{}
\rfoot{\thepage\ of \pageref{LastPage} }

\begin{document}

%%%%%%%%%%%%%%%%%%
%%%%%%%%%%%%%%%%%%
\title{Modules and Wedderburn Theory}
\author{Alexander Duncan}

\maketitle

Modern representation theory is usually phrased in the language of
non-commutative rings/algebras and their modules.
Until now I've avoided this language so I could focus on the basics.
However, the module-theoretic perspective is much more versatile --- even
for finite groups over $\mathbb{C}$.
It is much more urgently needed to discuss representation
theory over other fields. 

The main object is the \emph{group ring} $\mathbb{Z}G$
or \emph{group algebra} $kG$.  We will see that most of the language of
representation theory of finite groups can be efficiently expressed
as special cases of module theory over $kG$.
Then we will use the language to go even further.

Since module theory is part of the standard algebra qual sequence,
I will frequently omit proofs and even many definitions under the
assumption that you should have seen them before.
However, the emphasis in qual courses is usually on
\emph{commutative} rings so some reminders are in order.
I recommend finding a graduate algebra textbook with a good treatment of
modules, such as \cite[\S{10}]{DF} or \cite[\S{III}]{Lang},
to refer to if something is unfamiliar.

The ``basics'' of module-theoretic representation theory is in
almost every text I've referenced, often at a very early stage; 
see \cite{DF}, \cite{Etingof}, \cite{FultonHarris}, \cite{Lang} etc.
Serre's book \cite{Serre}, does not use the module-theoretic
perspective at first, but then abruptly changes gears in chapter 6 and
assumes you've seen Wedderburn theory (developed below).
I will also draw heavily from \cite[\S{12,13}]{AlperinBell},
\cite{AlperinLRT}, and \cite{CurtisReiner}.

%%%%%%%%%%%%%%%%%%
%%%%%%%%%%%%%%%%%%

\bibliographystyle{alpha}
\bibliography{rep_theory}

\end{document}


