\documentclass[12pt]{article}

\usepackage{amssymb,amsmath,amsthm}

% theorem styles (I like everything to have the same counter)
\theoremstyle{plain}
\newtheorem{theorem}{Theorem}[section]
\newtheorem{lemma}[theorem]{Lemma}
\newtheorem{proposition}[theorem]{Proposition}
\newtheorem{corollary}[theorem]{Corollary}
\newtheorem{conjecture}[theorem]{Conjecture}
\newtheorem{exercise}[theorem]{Exercise}
\theoremstyle{definition}
\newtheorem{definition}[theorem]{Definition}
\newtheorem{example}[theorem]{Example}
\theoremstyle{remark}
\newtheorem{remark}[theorem]{Remark}

% some numbering settings
\numberwithin{equation}{section}

\usepackage{fancyhdr}
\usepackage{lastpage}
\setlength{\headheight}{15.2pt}
\renewcommand{\footrulewidth}{0.4pt}% default is 0pt
\setlength{\footskip}{30pt}
\pagestyle{fancy}

\makeatletter
\let\ps@plain\ps@fancy 
\makeatother

\lhead{MATH 742}
\chead{Character Theory}
\rhead{Spring 2023}
\lfoot{Last Revised: \today}
\cfoot{}
\rfoot{\thepage\ of \pageref{LastPage} }

\begin{document}

%%%%%%%%%%%%%%%%%%
%%%%%%%%%%%%%%%%%%
\title{Character Theory}
\author{Alexander Duncan}

\maketitle

Some of the content here can be found in
\S 14--15 of \cite{AlperinBell} and
\S 2 of \cite{Serre}.
TODO

Throughout this document, we make the standing assumptions that $G$
is a finite group and $k$ is a field of characteristic coprime to the
order of $G$.
This allows us to appeal to Maschke's theorem:
every finite-dimensional representation is completely reducible.
Thus, every indecomposable representation is also irreducible.

\section{Representation Ring}

Let $R^+_k(G)$ be the set of isomorphism classes of
finite-dimensional representations of $G$ over $k$.
Given representations $V$ and $W$, let $[V]$ and $[W]$ denote their
images in $R^+_k(G)$.
We define addition on $R^+_k(G)$ via
\[ [V] + [W] = [V \oplus W] \]
and multiplication on $R^+_k(G)$ via
\[ [V] \cdot [W] = [V \otimes W]. \]
We write $0$ for $[0]$ and $1$ for $[k]$.
(One should check that these operations are well-defined.)

For representations $U,V,W$, let $u=[U]$, $v=[V]$, and $w=[W]$.
Write $0 = [0]$ for the zero representation and $1=[k]$
for the trivial representation.
Now, standard isomorphisms give rise to properties of $R^+_k(G)$ as follows:
\begin{align*}
U \oplus V &\cong V \oplus U & u+v&=v+u\\
U \oplus (V \oplus W) &\cong (U \oplus V) \oplus W & u+(v+w)&=(u+v)+w\\
0 \oplus V &\cong V & 0+v&=v\\
V \oplus 0 &\cong V & v+0&=v\\
U \otimes V &\cong V \otimes U & uv&=vu\\
U \otimes (V \otimes W) &\cong (U \otimes V) \otimes W & u(vw)&=(uv)w\\
k \otimes V &\cong V & 1v&=v\\
V \otimes k &\cong V & v1&=v\\
0 \otimes V &\cong 0 & 0v&=0\\
V \otimes 0 &\cong 0 & v0&=0\\
U \otimes (V \oplus W) &\cong (U \otimes V) \oplus (U \otimes W) &
u(v+w)&=uv+uw\\
(V \oplus W) \otimes U &\cong (V \otimes U) \oplus (V \otimes U) &
(v+w)u&=vu+wu
\end{align*}

We would like to conclude that $R^+_k(G)$ is a ring,
but we are missing additive inverses.  Indeed, no non-zero
representation has an additive inverse for dimensional reasons.
However, $R^+_k(G)$ does have the structure of a commutative \emph{rig}
(``a riNg without Negatives'').
In particular, $R^+_k(G)$ is a commutative monoid under addition,
and the non-zero elements of $R^+_k(G)$ form a commutative monoid under
multiplication.

Let $S_k(G)$ be the set of isomorphism classes of
irreducible representations of $G$ over $k$.
From the Krull-Schmidt theorem, every element $[V]$ of $R^+_G(k)$
can be written uniquely as a finite linear combination
\[
[V] = m_1 [W_1] + \cdots + m_r [W_r]
\]
where $[W_1],\ldots,[W_r] \in S_k(G)$ and $m_1,\ldots, m_r$ are
non-negative integers.
Thus there is an additive monoid isomorphism
$R^+_G(k) \cong \mathbb{N}^{\oplus S_k(G)}$ .
(We will see that $S_k(G)$ is always finite, but we distinguish between
the direct sum rather and the direct product until we prove this).

However, while the additive structure on $R^+_G(k)$ is very nice,
the \emph{multiplicative} structure is considerably more subtle.
Let $X$ be an indexing set for $S_k(G) = \{[W_i]\}_{i \in X}$.
The multiplicative structure is completely determined by the
non-negative integers $c^{ij}_\ell$ in the expressions
\[
[W_i] \cdot [W_j] = \sum_{\ell \in X} c^{ij}_\ell [W_\ell]
\]
where $i,j$ vary independently over all values of $X$.

\begin{remark}
The numbers $c^{ij}_\ell$ are called the \emph{Clebsch-Gordan numbers}
for the group $G$.  After selecting bases for the various $W_i$,
we have explicit matrices defining the isomorphism between
$W_i \otimes W_j$ and the decomposition into irreducible representations;
the entries of this matrix are called the \emph{Clebsch-Gordan
coefficients}.  The case where $G$ is the (infinite) special orthogonal
group $SO(3)$ is of special interest in quantum physics and quantum
chemistry.
\end{remark}

\subsection{Virtual Representations}

A \emph{virtual representation} $X$ is an element of the free abelian
group on $S_k(G)$.
More concretely, there is an expression
\[
X = m_1 [W_1] + m_2 [W_2] + \cdots + m_r [W_r]
\]
where $[W_1],\ldots,[W_r]$ are distinct isomorphism classes of
irreducible representations
and $m_1,\ldots, m_r$ are integers.
If all the elements $m_1,\ldots,m_r$ are non-negative,
then the virtual representation is an (isomorphism class of) a
representation via the evident interpretation in $R^+_k(G)$.

\begin{definition}
The \emph{representation ring $R_k(G)$ of $G$ over $k$} is
the additive group of virtual representations with multiplication
obtained by extending the multiplication of $R^+_k(G)$ by linearity.
\end{definition}

\begin{example}
If $G$ is the trivial group, then representations are just vector spaces.
In this case, $R^+_k(G) \cong \mathbb{N}$ where $V \mapsto \dim_k(V)$.
The ring of virtual representations is $R_k(G) \cong \mathbb{Z}$.
\end{example}

\begin{remark}
There are way more theorems about \emph{rings} than \emph{rigs}.
Thus, we ``adjoined negatives'' to make our rig $R^+_k(G)$
into a ring $R_k(G)$.
This procedure is an additive analog to forming the ring of fractions of a ring.
In general, one can similarly construct the \emph{Grothendieck group}
or \emph{group completion} $G$ from a commutative monoid $M$.
If the monoid $M$ is a rig, then the group $G$ becomes a ring.
However, we only get a nice embedding $M \hookrightarrow G$ if
$M$ is a \emph{cancellative} monoid.
\end{remark}

\section{Characters}

In this section, we assume throughout that $k=\mathbb{C}$.
Here for $z \in \mathbb{C}$, the expression $\overline{z}$ denotes the
complex conjugate.  Observe that if $z$ is a root of unity, then
$\overline{z}=z^{-1}$.  This small observation will have important
consequences.

\begin{definition}
Let $G$ be a finite group and $(V,\rho)$ a representation of $G$.
We define the \emph{character of $\rho$} as the function
\[
\chi : G \to \mathbb{C}
\]
given by $\chi(g)=\operatorname{tr}(\rho_g)$ for all $g \in G$.
\end{definition}

Characters satisfy numerous useful properties.


\begin{proposition}
Let $(V,\rho)$ and $(W,\sigma)$ be representations of a finite group
$G$ with characters $\phi$ and $\psi$, respectively.
Then:
\begin{enumerate}
\item $\phi(1)=\dim(V)$ where $1$ is identity of $G$,
\item $\phi(g^{-1}) = \overline{\phi(g)}$
for any $g \in G$, and
\item $\phi(hgh^{-1})=\phi(g)$, for all $g,h \in G$.
\item
the character $\chi$ of $V \oplus W$ is given by
\[
\chi(g) = \phi(g) + \psi(g) \textrm{ for } g \in G \ .
\]
\item
the character $\chi$ of $V \otimes W$ is given by
\[
\chi(g) = \phi(g) \psi(g) \textrm{ for } g \in G \ .
\]
\item
the character $\chi$ of $V^\vee$ is given by
\[
\chi(g) = \overline{\phi(g)} \textrm{ for } g \in G \ .
\]
\item
the character $\chi$ of $\operatorname{Hom}_k(V,W)$ is given by
\[
\chi(g) = \overline{\phi(g)} \psi(g) \textrm{ for } g \in G \ .
\]
\end{enumerate}
\end{proposition}


\begin{proof}
Fix an element $g \in G$.
Since $\rho(g)$ and $\sigma(g)$ are complex matrices of finite order,
they are both diagonalizable with eigenvalues roots of unity.
Let $e_1,\ldots,e_n$ be a diagonal basis for $\rho(g)$
with eigenvalues $\lambda_1, \ldots, \lambda_n$, and
let $f_1, \ldots, f_m$ be a diagonal basis for $\sigma(g)$
with eigenvalues $\mu_1, \ldots, \mu_m$.
In particular,
\[
\phi(g) = \lambda_1 + \cdots + \lambda_n \textrm{ and }
\psi(g) = \mu_1 + \cdots + \mu_m.
\]
We now prove each part of the statement

(1.)
$\phi(1)$ is the $n \times n$ identity matrix where $n=\dim(V)$.


(2.)
Recall $\zeta^{-1}=\overline{\zeta}$ for any root of unity
$\zeta$.
Thus
\[
\rho_{g^{-1}}(e_i)=\lambda_i^{-1}=\overline{\lambda_i}=\overline{\rho_g(e_i)}
\]
for each $g \in G$ and $i \in \{1,\ldots,n\}$.
Since the trace is simply the sum of the $\lambda_i$, the result
follows.

(3.) Recall that $\operatorname{tr}(AB)=\operatorname{tr}(BA)$ for any
matrices $A$ and $B$.  Thus
$\operatorname{tr}(ABA^{-1})=\operatorname{tr}(A^{-1}AB)=\operatorname{tr}(B)$.



(4.) Follows from the fact that $e_1, \ldots, e_n, f_1, \ldots, f_m$ is a
diagonal basis for $(\rho \oplus \sigma)(g)$ and that
\[
(\lambda_1 + \cdots + \lambda_n) + (\mu_1 + \cdots + \mu_m) = 
(\lambda_1 + \cdots + \lambda_n + \mu_1 + \cdots + \mu_m) \ .
\]



(5.) Note that $\{ e_i \otimes f_j \}$ is a diagonal basis for
$(\rho \otimes \sigma)(g)$.  The result follows from the observation
that
\[
(\lambda_1 + \cdots + \lambda_n) (\mu_1 + \cdots + \mu_m)
= \sum_{i=1}^n \sum_{j=1}^m \lambda_i \mu_j \ .
\]



(6.) Denote by $e_1^\vee, \ldots, e_n^\vee$ the dual basis of $e_1,
\ldots, e_n$.
Observe that
\[
\rho^\vee(g)(e_i^\vee)(e_j)=e_i^\vee(\rho(g^{-1})(e_j))
= \lambda_i^{-1}\delta_{ij}=\overline{\lambda_i}\delta_{ij} \ .
\]



(7.) Follows from the isomorphism
$\operatorname{Hom}_k(\rho,\sigma) \cong \rho^\vee \otimes \sigma$.
\end{proof}


\begin{exercise}
The character $\chi$ of $\operatorname{Sym}^2(V) \cong S^2(V)$ is given by
\[
\chi(g) = \frac{1}{2}\left(\phi(g)^2 + \phi(g^2)\right)
\textrm{ for } g \in G \ .
\]
The character $\chi$ of $\operatorname{Alt}^2(V) \cong \Lambda^2(V)$ is given by
\[
\chi(g) = \frac{1}{2}\left(\phi(g)^2 - \phi(g^2)\right)
\textrm{ for } g \in G \ .
\]
\end{exercise}

The third property above proves that all characters are \emph{class
functions}:

\begin{definition}
A function $f : G \to \mathbb{C}$ is called a \emph{class function}
if $f(hgh^{-1})=f(g)$ for all $g,h \in G$.
The set of all class functions is denoted $C(G)$.
\end{definition}

Equivalently, $C(G)$ consists of exactly those functions $f : G \to
\mathbb{C}$ such that $f(g)=f(h)$ whenever $g, h$ belong to the
same conjugacy class in $G$.

The set of all functions $f : G \to \mathbb{C}$ is a
$\mathbb{C}$-algebra via pointwise addition and multiplication.
Alternatively, the ring is the same as the usual ring structure on the
Cartesian power $\mathbb{C}^G$.
The subset of class function $C(G)$ are a subalgebra of this ring.
If $c(G)$ denotes the number of conjugacy classes of $G$,
then $C(G)$ is naturally isomorphic to $\mathbb{C}^{c(G)}$.

\begin{example}
Let $G$ be a finite group and let $\rho$ be the regular representation.
The character $\chi$ of $\rho$ satisfies
\[
\chi(g) = \begin{cases}
|G| & \textrm{for $g=1$},\\
0 & \textrm{otherwise}
\end{cases} .
\]
\end{example}

The properties proven above about characters imply the following nice
corollary:

\begin{corollary}
The assignment of a representation to its character
induces a ring homomorphism
\[
\Omega: R(G) \to C(G)
\]
from the representation ring to the algebra of class functions on $G$.
\end{corollary}

We will see shortly that $\Omega$ is \emph{injective}.
An element in the image of the map $\Omega$ above is called a \emph{virtual
character}.  Note that $R(G)$ is only a ring, while $C(G)$ is a
$\mathbb{C}$-algebra.  They are never isomorphic, but we will see that
$R(G) \otimes_\mathbb{Z} \mathbb{C} \cong C(G)$ as $\mathbb{C}$-algebras.

Since the set of isomorphism classes of irreducible representations of
$G$ form a $\mathbb{Z}$-basis for $R(G)$
and the set of conjugacy classes of $G$
form a $\mathbb{C}$-basis for $C(G)$,
the map $\Omega$ is totally determined by pairing these two basis.
The resulting matrix is called the \emph{character table} of $G$.

\begin{example}
It turns out the symmetric group $S_4$ has exactly $5$ irreducible
representations $W_1, \ldots, W_5$.
The character table is as follows:
\begin{center}
\begin{tabular}{|c|c|c|c|c|c|}
\hline 
 & $e $ & $(12)$ & $(12)(34)$ & $(123)$ & $(1234)$\\
\hline 
\hline 
$W_1$ & $1$ & $1$ & $1$ & $1$ & $1$\\
\hline 
$W_2$ & $1$ & $-1$ & $1$ & $1$ & $-1$\\
\hline 
$W_3$ & $2$ & $0$ & $2$ & $-1$ & $0$\\
\hline 
$W_4$ & $3$ & $1$ & $-1$ & $0$ & $-1$\\
\hline 
$W_5$ & $3$ & $-1$ & $-1$ & $0$ & $1$\\
\hline 
\end{tabular}
\end{center}
\end{example}

There are some common conventions regarding character tables.
Almost universally,
the trivial representation is the first row and the class of the
identity is the first column.
Another convention is that the irreducible representations are listed in
increasing order by dimension.
Finally, the conjugacy classes are usually listed in increasing order by
the order of a representing element.

Computer algebra packages can produce (or look up) character tables
for finite groups.  The orderings of the rows and the columns of the
table may not be consistent even if exactly the same program is run on
exactly the same data!

%%%%%%%%%%%%%%%%%%
%%%%%%%%%%%%%%%%%%

\bibliographystyle{alpha}
\bibliography{rep_theory}

\end{document}


