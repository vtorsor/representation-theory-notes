\documentclass[12pt]{article}

\usepackage{amssymb,amsmath,amsthm}

% theorem styles (I like everything to have the same counter)
\theoremstyle{plain}
\newtheorem{theorem}{Theorem}[section]
\newtheorem{lemma}[theorem]{Lemma}
\newtheorem{proposition}[theorem]{Proposition}
\newtheorem{corollary}[theorem]{Corollary}
\newtheorem{conjecture}[theorem]{Conjecture}
\newtheorem{exercise}[theorem]{Exercise}
\theoremstyle{definition}
\newtheorem{definition}[theorem]{Definition}
\newtheorem{example}[theorem]{Example}
\theoremstyle{remark}
\newtheorem{remark}[theorem]{Remark}

% some numbering settings
\numberwithin{equation}{section}

\usepackage{fancyhdr}
\usepackage{lastpage}
\setlength{\headheight}{15.2pt}
\renewcommand{\footrulewidth}{0.4pt}% default is 0pt
\setlength{\footskip}{30pt}
\pagestyle{fancy}

\makeatletter
\let\ps@plain\ps@fancy 
\makeatother

\lhead{MATH 742}
\chead{Character Theory}
\rhead{Spring 2023}
\lfoot{Last Revised: \today}
\cfoot{}
\rfoot{\thepage\ of \pageref{LastPage} }

\begin{document}

%%%%%%%%%%%%%%%%%%
%%%%%%%%%%%%%%%%%%
\title{Character Theory}
\author{Alexander Duncan}

\maketitle

{\bf
Throughout this document, we make the standing assumptions that $G$
is a finite group and $k$ is a field of characteristic coprime to the
order of $G$.}
This allows us to appeal to Maschke's theorem:
every finite-dimensional representation is completely reducible.
Thus, every indecomposable representation is also irreducible.

Character theory is the core of finite group representations.
Much of the material below can be found in, for example,
\cite[\S{14--15}]{AlperinBell},
\cite[\S{18.3}]{DF},
\cite[\S{4}]{Etingof},
\cite[\S{2}]{FultonHarris},
\cite[\S{XVIII.1--5}]{Lang}, or
\cite[\S{2}]{Serre}.

\section{Representation Ring}

Let $R^+_k(G)$ be the set of isomorphism classes of
finite-dimensional representations of $G$ over $k$.
Given representations $V$ and $W$, let $[V]$ and $[W]$ denote their
images in $R^+_k(G)$.
We define addition on $R^+_k(G)$ via
\[ [V] + [W] := [V \oplus W] \]
and multiplication on $R^+_k(G)$ via
\[ [V] \cdot [W] := [V \otimes W]. \]
We write $0$ for $[0]$ and $1$ for $[k]$.
(One should check that these operations are well-defined.)

For representations $U,V,W$, let $u=[U]$, $v=[V]$, and $w=[W]$.
Write $0 = [0]$ for the zero representation and $1=[k]$
for the trivial representation.
Now, standard isomorphisms give rise to properties of $R^+_k(G)$ as follows:
\begin{align*}
U \oplus V &\cong V \oplus U & u+v&=v+u\\
U \oplus (V \oplus W) &\cong (U \oplus V) \oplus W & u+(v+w)&=(u+v)+w\\
0 \oplus V &\cong V & 0+v&=v\\
V \oplus 0 &\cong V & v+0&=v\\
U \otimes V &\cong V \otimes U & uv&=vu\\
U \otimes (V \otimes W) &\cong (U \otimes V) \otimes W & u(vw)&=(uv)w\\
k \otimes V &\cong V & 1v&=v\\
V \otimes k &\cong V & v1&=v\\
0 \otimes V &\cong 0 & 0v&=0\\
V \otimes 0 &\cong 0 & v0&=0\\
U \otimes (V \oplus W) &\cong (U \otimes V) \oplus (U \otimes W) &
u(v+w)&=uv+uw\\
(V \oplus W) \otimes U &\cong (V \otimes U) \oplus (V \otimes U) &
(v+w)u&=vu+wu
\end{align*}

We would like to conclude that $R^+_k(G)$ is a ring,
but we are missing additive inverses.  Indeed, no non-zero
representation has an additive inverse for dimensional reasons.
However, $R^+_k(G)$ does have the structure of a commutative \emph{rig}
(``a riNg without Negatives'').
In particular, $R^+_k(G)$ is a commutative monoid under addition,
and the non-zero elements of $R^+_k(G)$ form a commutative monoid under
multiplication.

Let $S_k(G)$ be the set of isomorphism classes of
irreducible representations of $G$ over $k$.
From the Krull-Schmidt theorem, every element $[V]$ of $R^+_G(k)$
can be written uniquely as a finite linear combination
\[
[V] = m_1 [W_1] + \cdots + m_r [W_r]
\]
where $[W_1],\ldots,[W_r] \in S_k(G)$ and $m_1,\ldots, m_r$ are
non-negative integers.
Thus there is an additive monoid isomorphism
$R^+_G(k) \cong \mathbb{N}^{\oplus S_k(G)}$ .
(We will see that $S_k(G)$ is always finite, but we distinguish between
the direct sum rather and the direct product until we prove this).

However, while the additive structure on $R^+_G(k)$ is very nice,
the \emph{multiplicative} structure is considerably more subtle.
Let $X$ be an indexing set for $S_k(G) = \{[W_i]\}_{i \in X}$.
The multiplicative structure is completely determined by the
non-negative integers $c^{ij}_\ell$ in the expressions
\[
[W_i] \cdot [W_j] = \sum_{\ell \in X} c^{ij}_\ell [W_\ell]
\]
where $i,j$ vary independently over all values of $X$.

\begin{remark}
The numbers $c^{ij}_\ell$ are called the \emph{Clebsch-Gordan numbers}
for the group $G$.  After selecting bases for the various $W_i$,
we have explicit matrices defining the isomorphism between
$W_i \otimes W_j$ and the decomposition into irreducible representations;
the entries of this matrix are called the \emph{Clebsch-Gordan
coefficients}.  Analogs for infinite groups like the special orthogonal
group $SO(3)$ is of special interest in quantum physics and quantum
chemistry.
\end{remark}

\subsection{Virtual Representations}

A \emph{virtual representation} $X$ is an element of the free abelian
group on $S_k(G)$.
More concretely, there is an expression
\[
X = m_1 [W_1] + m_2 [W_2] + \cdots + m_r [W_r]
\]
where $[W_1],\ldots,[W_r]$ are distinct isomorphism classes of
irreducible representations
and $m_1,\ldots, m_r$ are integers.
If all the elements $m_1,\ldots,m_r$ are non-negative,
then the virtual representation is an (isomorphism class of) a
representation via the evident interpretation in $R^+_k(G)$.

\begin{definition}
The \emph{representation ring $R_k(G)$ of $G$ over $k$} is
the additive group of virtual representations with multiplication
obtained by extending the multiplication of $R^+_k(G)$ by linearity.
If $k=\mathbb{C}$, then we simply write $R(G)$.
\end{definition}

\begin{example}
If $G=1$ is the trivial group, then representations are just vector spaces.
In this case, $R^+_k(1) \cong \mathbb{N}$ where $V \mapsto \dim_k(V)$.
The ring of virtual representations is $R_k(1) \cong \mathbb{Z}$.
\end{example}

\begin{remark}
There are way more theorems about \emph{rings} than \emph{rigs}.
Thus, we ``adjoined negatives'' to make our rig $R^+_k(G)$
into a ring $R_k(G)$.
This procedure is an additive analog to forming the ring of fractions of a ring.
In general, one can similarly construct the \emph{Grothendieck group}
or \emph{group completion} $G$ from a commutative monoid $M$.
If the monoid $M$ is a rig, then the group $G$ becomes a ring.
However, we only get an embedding $M \hookrightarrow G$ if
$M$ is \emph{cancellative}.
\end{remark}

\section{Characters}

In this section, we assume throughout that $k=\mathbb{C}$.
Here for $z \in \mathbb{C}$, the expression $\overline{z}$ denotes the
complex conjugate.  Observe that if $z$ is a root of unity, then
$\overline{z}=z^{-1}$.  This small observation will have important
consequences.

\begin{definition}
Let $G$ be a finite group and $(V,\rho)$ a representation of $G$.
We define the \emph{character of $\rho$} as the function
\[
\chi_V : G \to \mathbb{C}
\]
given by $\chi_V(g)=\operatorname{tr}(\rho_g)$ for all $g \in G$. 
\end{definition}

Characters satisfy numerous useful properties.


\begin{proposition}
Let $(V,\rho)$ and $(W,\sigma)$ be representations of a finite group
$G$.
Then:
\begin{enumerate}
\item $\chi_V(1)=\dim(V)$ where $1$ is identity of $G$,
\item $\chi_V(g^{-1}) = \overline{\chi_V(g)}$
for any $g \in G$, and
\item $\chi_V(hgh^{-1})=\chi_V(g)$, for all $g,h \in G$.
\item
$\displaystyle
\chi_{V \oplus W}(g) = \chi_V(g) + \chi_W(g)$ for $g \in G$.
\item
$\displaystyle
\chi_{V \otimes W}(g) = \chi_V(g)\chi_W(g)$ for $g \in G$.
\item
$\displaystyle
\chi_{V^\vee}(g) =  \overline{\chi_V(g)}$ for $g \in G$.
\item
$\displaystyle
\chi_{\operatorname{Hom}_k(V,W)}(g) = \overline{\chi_V(g)} \chi_W(g)$ for $g \in G$.
\end{enumerate}
\end{proposition}


\begin{proof}
Fix an element $g \in G$.
Since $\rho(g)$ and $\sigma(g)$ are complex matrices of finite order,
they are both diagonalizable with eigenvalues roots of unity.
Let $e_1,\ldots,e_n$ be a diagonal basis for $\rho(g)$
with eigenvalues $\lambda_1, \ldots, \lambda_n$, and
let $f_1, \ldots, f_m$ be a diagonal basis for $\sigma(g)$
with eigenvalues $\mu_1, \ldots, \mu_m$.
In particular,
\[
\chi_V(g) = \lambda_1 + \cdots + \lambda_n \textrm{ and }
\chi_W(g) = \mu_1 + \cdots + \mu_m.
\]
We now prove each part of the statement

(1.)
$\chi_V(1)$ is the $n \times n$ identity matrix where $n=\dim(V)$.


(2.)
Recall $\zeta^{-1}=\overline{\zeta}$ for any root of unity
$\zeta$.
Thus
\[
\rho_{g^{-1}}(e_i)=\lambda_i^{-1}=\overline{\lambda_i}=\overline{\rho_g(e_i)}
\]
for each $g \in G$ and $i \in \{1,\ldots,n\}$.
Since the trace is simply the sum of the $\lambda_i$, the result
follows.

(3.) Recall that $\operatorname{tr}(AB)=\operatorname{tr}(BA)$ for any
matrices $A$ and $B$.  Thus
$\operatorname{tr}(ABA^{-1})=\operatorname{tr}(A^{-1}AB)=\operatorname{tr}(B)$.



(4.) Follows from the fact that $e_1, \ldots, e_n, f_1, \ldots, f_m$ is a
diagonal basis for $(\rho \oplus \sigma)(g)$ and that
\[
(\lambda_1 + \cdots + \lambda_n) + (\mu_1 + \cdots + \mu_m) = 
(\lambda_1 + \cdots + \lambda_n + \mu_1 + \cdots + \mu_m) \ .
\]



(5.) Note that $\{ e_i \otimes f_j \}$ is a diagonal basis for
$(\rho \otimes \sigma)(g)$.  The result follows from the observation
that
\[
(\lambda_1 + \cdots + \lambda_n) (\mu_1 + \cdots + \mu_m)
= \sum_{i=1}^n \sum_{j=1}^m \lambda_i \mu_j \ .
\]



(6.) Denote by $e_1^\vee, \ldots, e_n^\vee$ the dual basis of $e_1,
\ldots, e_n$.
Observe that
\[
\rho^\vee(g)(e_i^\vee)(e_j)=e_i^\vee(\rho(g^{-1})(e_j))
= \lambda_i^{-1}\delta_{ij}=\overline{\lambda_i}\delta_{ij} \ .
\]



(7.) Follows from the isomorphism
$\operatorname{Hom}_k(V,W) \cong V^\vee \otimes W$.
\end{proof}


\begin{exercise}
Using notation as the previous proposition.
\begin{enumerate}
\item $\displaystyle
\chi_{\operatorname{Sym}^2(V)}(g)=
\chi_{\mathcal{S}^2(V)}(g)
=  \frac{1}{2}\left(\chi(g)^2 + \chi(g^2)\right)$ for $g \in G$.
\item $\displaystyle
\chi_{\operatorname{Alt}^2(V)}(g)=
\chi_{\Lambda^2(V)}(g)
=  \frac{1}{2}\left(\chi(g)^2 - \chi(g^2)\right)$ for $g \in G$.
\end{enumerate}
\end{exercise}

The proposition above proves that all characters are \emph{class
functions}:

\begin{definition}
A function $f : G \to \mathbb{C}$ is called a \emph{class function}
if $f(hgh^{-1})=f(g)$ for all $g,h \in G$.
The set of all class functions is denoted $C(G)$.
\end{definition}

Equivalently, $C(G)$ consists of exactly those functions $f : G \to
\mathbb{C}$ such that $f(g)=f(h)$ whenever $g, h$ belong to the
same conjugacy class in $G$.

The set of all functions $f : G \to \mathbb{C}$ is a
$\mathbb{C}$-algebra via pointwise addition and multiplication.
Alternatively, the ring is the same as the usual ring structure on the
Cartesian power $\mathbb{C}^G$.
The subset of class function $C(G)$ are a subalgebra of this ring.
If $c(G)$ denotes the number of conjugacy classes of $G$,
then $C(G)$ is naturally isomorphic to $\mathbb{C}^{c(G)}$.

\begin{example} \label{eg:regular_rep}
Let $V$ be the regular representation.  Then
\[
\chi_V(g) = \begin{cases}
|G| & \textrm{for $g=1$},\\
0 & \textrm{otherwise}.
\end{cases}
\]
This follow since $he_g=e_g$ only when $h = 1$ in the natural basis $\{e_g\}_{g \in G}$.
\end{example}

The properties proven above about characters imply the following nice
corollary:

\begin{corollary}
The assignment of a representation to its character
induces a ring homomorphism
\[
\chi_{\bullet}: R(G) \to C(G)
\]
from the representation ring to the algebra of class functions on $G$.
\end{corollary}

We will see shortly that $\chi_{\bullet}$ is \emph{injective}.
An element in the image of the map $\chi_{\bullet}$ above is called a \emph{virtual
character}.  Note that $R(G)$ is only a ring, while $C(G)$ is a
$\mathbb{C}$-algebra.  They are never isomorphic, but we will see that
$R(G) \otimes_\mathbb{Z} \mathbb{C} \cong C(G)$ as $\mathbb{C}$-algebras.

Since the set of isomorphism classes of irreducible representations of
$G$ form a $\mathbb{Z}$-basis for $R(G)$
and the set of conjugacy classes of $G$
form a $\mathbb{C}$-basis for $C(G)$,
the map $\chi_{\bullet}$ is totally determined by pairing these two bases.
The resulting matrix is called the \emph{character table} of $G$.

\begin{example}
It turns out the symmetric group $S_4$ has exactly $5$ irreducible
representations $W_1, \ldots, W_5$.
The character table is as follows:
\begin{center}
\begin{tabular}{|c|c|c|c|c|c|}
\hline 
 & $e $ & $(12)$ & $(12)(34)$ & $(123)$ & $(1234)$\\
\hline 
\hline 
$W_1$ & $1$ & $1$ & $1$ & $1$ & $1$\\
\hline 
$W_2$ & $1$ & $-1$ & $1$ & $1$ & $-1$\\
\hline 
$W_3$ & $2$ & $0$ & $2$ & $-1$ & $0$\\
\hline 
$W_4$ & $3$ & $1$ & $-1$ & $0$ & $-1$\\
\hline 
$W_5$ & $3$ & $-1$ & $-1$ & $0$ & $1$\\
\hline 
\end{tabular}
\end{center}
\end{example}

There are some common conventions regarding character tables.
Almost universally,
the trivial representation is the first row and the class of the
identity is the first column.
Another convention is that the irreducible representations are listed in
increasing order by dimension.
Finally, the conjugacy classes are usually listed in increasing order by
the order of a representing element.

Computer algebra packages can produce (or look up) character tables
for finite groups.  Warning: the orderings of the rows and the columns of the
table may not be consistent even if exactly the same program is run on
exactly the same data!

\section{Orthogonality Relations}

We begin by pointing out that class functions give rise to
$G$-equivariant endomorphisms.

\begin{lemma} \label{lem:class_function_endo}
Let $(V,\rho)$ be a complex representation of a finite group $G$,
let $f : G \to \mathbb{C}$ be a function, and let $F \in
\operatorname{End}_k(V)$ be the endomorphism defined by
\[
F(v) = \sum_{g \in G} f(g)\rho_g(v) .
\]
Then $F$ is $G$-equivariant if and only if $f$ is a class function.
\end{lemma}

\begin{proof}
Exercise.  This is the same reindexing argument as in Maschke's Theorem.
\end{proof}

Recall that if $(V,\rho)$ is a linear representation of a finite group $G$,
then $V^G$ is the invariant subspace of vectors
$v \in V$ such that $\rho_g(v)=v$ for all $g \in G$.

\begin{lemma} \label{lem:proj_onto_invariants}
If $(V,\rho)$ be a linear representation of a finite group $G$,
then
\[ \dim(V^G) = \frac{1}{|G|} \sum_{g \in G} \chi_V(g). \]
\end{lemma}

\begin{proof}
By the same argument as in the proof of Maschke's Theorem,
the linear transformation $\pi : V \to V$ defined by
\[
\pi(v) = \frac{1}{|G|} \sum_{g \in G} \rho_g(v)
\]
is a $G$-equivariant projection onto $V^G$.
Choosing an appropriate basis for $V$, the map $\pi$ corresponds to a
matrix with $\dim(V^G)$ ones on the diagonal and zeroes elsewhere.
The computation
\[
\frac{1}{|G|} \sum_{g \in G} \chi_V(g)
= \frac{1}{|G|} \sum_{g \in G} \operatorname{tr}( \rho(g) )
= \operatorname{tr}\left( \frac{1}{|G|} \sum_{g \in G}\rho(g) \right)
= \operatorname{tr}( \pi ) = \dim(V^G)
\]
finishes the argument.
\end{proof}

\begin{definition} \label{def:innerProduct}
Define a pairing
\[
( \phi , \psi ) := \frac{1}{|G|}
\sum_{g \in G} \overline{\phi(g)} \psi(g)
\]
for any two functions $\phi, \psi : G \to \mathbb{C}$.
\end{definition}

\begin{proposition}
The product $(-,-)$ makes $C(G)$ an inner product space.
\end{proposition}

\begin{proof}
The pairing $(-,-)$ is conjugate linear in the first variable and linear in
the second.  Also, $(\psi,\psi)$ is a sum of non-negative real numbers
which is zero if and only if $\psi(g)=0$ for every $g \in G$.
\end{proof}


The inner product has an important interpretation:

\begin{lemma}
$\displaystyle
(\chi_V, \chi_W) = \dim_k\left( \operatorname{Hom}^G_k\left(V,
W\right)\right)$.
\end{lemma}

\begin{proof}
If $\tau$ is the character of $\operatorname{Hom}_k(V,W)$, then
\[ ( \chi_V , \chi_W ) = \frac{1}{|G|}
\sum_{g \in G} \overline{\chi_V(g)} \chi_W(g)
= \frac{1}{|G|} \sum_{g \in G} \tau(g) .
\]
Thus:
\[ ( \chi_V , \chi_W ) =
\dim_k\left(\left( \operatorname{Hom}_k\left(V,W\right) \right)^G\right) =
\dim_k( \operatorname{Hom}^G_k(V,W) ) \]
as desired.
\end{proof}

A character is \emph{irreducible} if its corresponding representation
is.
The main result of this section is as follows:

\begin{theorem}
The irreducible characters are an orthonormal basis for the space of
class functions $C(G)$.
\end{theorem}

\begin{proof}
Suppose $V$ and $W$ are irreducible.
Then Schur's Lemma gives us
\[
\left( \chi_V, \chi_W \right)
= \dim_k \operatorname{Hom}^G_k(V,W) =
\begin{cases}
1 & V \cong W\\
0 & V \not\cong W
\end{cases}.
\]
Thus, the irreducible characters are an orthonormal set.
It remains to show they span $C(G)$.

Let $f \in C(G)$ be orthogonal to every irreducible character
(thus every character).
For any representation $(V,\rho)$, the endomorphism
\[
F_\rho = \sum_{g \in G} \overline{f(g)} \rho_g
\]
is equivariant by Lemma~\ref{lem:class_function_endo}.
Taking the trace, we find
\[
\operatorname{tr}(F_\rho) =
\sum_{g \in G} \overline{f(g)} \chi_V(g)
= |G| (f,\chi_V),
\]
which is $0$ by the orthogonality condition.
When $V$ is irreducible,
$F_\rho$ is scalar multiplication by Schur's Lemma;
moreover, the trace being zero implies that
$F_\rho=0$.
If $W$ is an irreducible subrepresentation of $V$,
then $F_\rho(W) \subseteq W$ since $F_\rho$ is a linear combination of
functions satisfying that property.
Thus, we conclude $F_\rho=0$ for all representations $\rho$.

When $V$ is the regular representation with basis $\{e_g\}_{g \in G}$,
we obtain
\[
0=F_\rho(e_1)=\sum_{g \in G} \overline{f(g)} e_g .
\]
We must conclude $f(g)=0$ for all $g \in G$.  Thus no class function is
orthogonal to every irreducible character.
\end{proof}

This theorem is enormously powerful.  In particular, it tells us that
the number of irreducible representations is not only \emph{finite}, but
exactly how many there are:

\begin{corollary}
The number of irreducible representations is equal to the number of
conjugacy classes in $G$.
\end{corollary}

The character table can be viewed as the ``change of basis'' matrix
between irreducible representations of $R(G)$ and characteristic
functions of conjugacy classes in $C(G)$.
While there exists a bijection between conjugacy classes
and irreducible representations, we do not explicitly construct one.
Indeed, there are good reasons to not expect an explicit ``combinatorial
bijection'' in general. 

\begin{corollary}
If $V$ is a representation with isotypic decomposition
\[
V \cong W_1^{\oplus m_1} \oplus \cdots \oplus W_r^{\oplus m_r}
\]
for irreducible representations $W_1, \ldots, W_r$,
then we may recover the multiplicities via
\[
m_i = \left( \chi_V, \chi_{W_i} \right).
\]
\end{corollary}

\begin{proof}
Exercise.
\end{proof}

\begin{corollary}
A character $\chi$ is irreducible if and only if $(\chi,\chi)=1$.
\end{corollary}

\begin{proof}
Exercise.
\end{proof}

\begin{corollary}
Suppose $G$ has irreducible representations $V_1,\ldots,V_r$
with dimensions $n_1,\ldots, n_r$.  Then the isotypic decomposition of
the regular representation is
\[
V_1^{n_1} \oplus \cdots \oplus V_r^{n_r} .
\]
In particular, $|G|=n_1^2 + n_2^2 + \cdots + n_r^2$.
\end{corollary}

\begin{proof}
Let $W$ be the regular representation.
Using Example~\ref{eg:regular_rep}, we compute
\[
( \chi_{V_i} , \chi_W ) = \frac{1}{|G|}
\sum_{g \in G} \overline{\chi_{V_i}(g)} \chi_W(g)
= \frac{1}{|G|} \overline{\chi_{V_i}(1)} |G|
= \dim(V_i)=n_i.
\]
The second statement follows by comparing the dimensions
in the regular representation with its isotypic decomposition.
\end{proof}

\subsection{Computing inner products}

In practice, it is inefficient to compute inner products by summing over
every element in the group.  We already know that the class functions
will give the same output for each element of the conjugacy class.
Therefore, it is convenient to rewrite the inner product to take
advantage of this.

\begin{proposition}
Let $K_1, \ldots, K_r$ be the conjugacy classes of $G$
and let $g_1,\ldots, g_r$ be representatives in each class.
Then
\[
( \psi, \phi ) = \sum_{i=1}^r \frac{|K_i|}{|G|}
\overline{\psi(g_i)}\phi(g_i)
\]
for all class functions $\psi,\phi \in C(G)$.
\end{proposition}

Recall that an orthogonal matrix has an orthogonal transpose.
The formula above gives an orthonormal pairing for the
\emph{rows} of the character table (when appropriately weighted by the number of elements in
each conjugacy class).  We also have orthogonality of the
\emph{columns} via the following:

\begin{exercise}
Let $\chi_1,\ldots,\chi_r$ be the irreducible characters.
Let $K(g)$ denote the number of elements in the conjugacy class of $g \in
G$.
We have
\[
\sum_{i=1}^r \overline{\chi_i(g)} \chi_i(h) =
\begin{cases}
\frac{|G|}{K(g)} & \textrm{if $g$ is conjugate to $h$}\\
0 & \textrm{otherwise}
\end{cases}
\]
for $g, h \in G$.
\end{exercise}

\subsection{Explicit Decomposition}

If we know the characters, then we can explicitly compute the isotypic
decomposition of linear representation. 

\begin{theorem}
Let $W_1,\ldots, W_r$ be the irreducible representations of $G$ with
corresponding characters $\chi_1,\ldots, \chi_r$.
Let $(V,\rho)$ be a representation and let $V_i$ be the isotypic
component of $V$ associated to $W_i$.
Then the map $\pi_i : V \to V$ given by
\[
\pi_i(v) := \frac{\chi_i(1)}{|G|} \sum_{g \in G} \overline{\chi_i(g)} \rho_g(v)
\]
is the projection onto the isotypic component $V_i$.
\end{theorem}

\begin{proof}
From Lemma~\ref{lem:class_function_endo}, we see that
$\pi_i$ is equivariant.
Thus $\pi_i(U) \subseteq U$ for every subrepresentation $U$ of $V$.
By Schur's Lemma, $\pi_i|_U$ is scalar multiplication whenever
$U$ is irreducible.
If $\psi$ is the character of an irreducible subrepresentation $U$ of $V$,
then
\[
\operatorname{tr}(\pi_i|_U)=\chi_i(1) ( \chi_i, \psi ) .
\]
We conclude that
\[
\pi_i|_U = \frac{\chi_i(1)}{\psi(1)} ( \chi_i, \psi ) \ .
\]
Thus $\pi_i|_U=0$ if $W_i \not\cong U$ and $\pi_i|_U=\operatorname{id}$
if $W_i \cong U$.
\end{proof}

Note that the theorem above generalizes the projection constructed in
the proof of Lemma~\ref{lem:proj_onto_invariants}.

There are also formulas that can explicitly decompose an
isotypic representation into irreducible summands.
Such a decomposition obviously cannot be canonical.
See \S 2.7 of \cite{Serre}.

\newpage

\section{Examples}

\begin{example}
Let $G = \mathbb{Z}/n\mathbb{Z}$.  Let $\zeta = e^{2\pi i/n}$
be the principal primitive $n$th root of unity.
The conjugacy classes of $G$ are just the elements of $G$.
The irreducible characters are just
the group homomorphisms $G \to \mathbb{C}^\times$.
They are precisely the maps
\[
\chi_i(j) = \zeta^{ij}
\]
for $i \in 1,\ldots,n$ and $j \in G$.
The character table is as follows:
\begin{center}
\begin{tabular}{|c|c|c|c|c|c|}
\hline 
 & $0 $ & $1$ & $2$ & $\cdots$ & ${n-1}$\\
\hline 
\hline 
$\chi_0$ & $1$ & $1$ & $1$ & $\cdots$ & $1$\\
\hline 
$\chi_1$ & $1$ & $\zeta$ & $\zeta^2$ & $\cdots$ & $\zeta^{-1}$\\
\hline 
$\chi_2$ & $1$ & $\zeta^2$ & $\zeta^4$ & $\cdots$ & $\zeta^{-2}$\\
\hline 
$\vdots$ & $\vdots$ & $\vdots$ & $\vdots$ & $\ddots$ & $\vdots$\\
\hline 
$\chi_{n-1}$ & $1$ & $\zeta^{-1}$ & $\zeta^{-2}$ & $\cdots$ & $\zeta$\\
\hline
\end{tabular}
\end{center}
This is exactly the \emph{discrete Fourier transform}.
\end{example}

\begin{example}
More generally, if $G$ is a finite abelian group, then
the character table is $|G| \times |G|$ since every conjugacy class has
only one element.
Once again, the irreducible characters are group homomorphisms
$G \to \mathbb{C}^\times$.
For example, in the case of the Klein 4-group we have:
\begin{center}
\begin{tabular}{|c|c|c|c|c|}
\hline 
 & $(0,0)$ & $(1,0)$ & $(0,1)$ & $(1,1)$\\
\hline 
\hline 
$\chi_{(0,0)}$ & $1$ & $1$ & $1$ & $1$\\
\hline 
$\chi_{(1,0)}$ & $1$ & $-1$ & $1$ & $-1$\\
\hline 
$\chi_{(0,1)}$ & $1$ & $1$ & $-1$ & $-1$\\
\hline 
$\chi_{(1,1)}$ & $1$ & $-1$ & $-1$ & $1$\\
\hline
\end{tabular}
\end{center}
\end{example}

\begin{example}
Let $G$ be the dihedral group of order $2n$:
\[
D_{2n} = \langle s,r \mid s^2, r^n, (sr)^2 \rangle
\]
where $n \ge 3$.

Let $\zeta$ be a primitive $n$th root of unity.
Aside from the trivial representation $\rho_0$,
there is always a $1$-dimensional representation $\rho_1$
where $\rho_1(s)=(-1)$ and $\rho_1(r)=(1)$.
For each integer $1 \le i \le n-1$ we may produce a two-dimensional
representation $\sigma_i$ as follows:
\[
\sigma_i(s) = \begin{pmatrix} 0&1\\1&0 \end{pmatrix} \quad
\sigma_i(r) = \begin{pmatrix} \zeta^i&0\\0&\zeta^{-i} \end{pmatrix} .
\]
However, notice that $\sigma_i$ is isomorphic to $\sigma_{n-i}$
and, in the case where $n$ is even, we see that $\sigma_{n/2}$ is not
irreducible.
Thus, when $n$ is even, we also have two additional
$1$-dimensional representations
$\chi_2$ and $\chi_3$.

Recall that $\zeta^i+\zeta^{-i}=2\cos(\frac{2\pi i}{n})$.
It's natural to consider the character tables in the even and odd
cases separately.
When $n$ is odd we have:
\begin{center}
\begin{tabular}{|c|c|c|c|c|c|}
\hline 
 & $1$ & $sr^i$ & $r^{\pm 1}$ & $\cdots$ & $r^{\pm (n-1)/2}$\\
\hline 
\hline 
$\chi_0$ & $1$ & $1$ & $1$ & $\cdots$ & $1$\\
\hline 
$\chi_1$ & $1$ & $-1$ & $1$ & $\cdots$ & $1$\\
\hline 
$\sigma_{1}$ & $2$ & $0$ & $2\cos\left(\frac{2\pi}{n}\right)$ & $\cdots$
& $2\cos\left(\frac{ (n-1) \pi}{n}\right)$\\
\hline 
$\vdots$ & $\vdots$ & $\vdots$ & $\vdots$ & $\ddots$ & $\vdots$\\
\hline
$\sigma_{(n-1)/2}$ & $2$ & $0$ & $2\cos\left(\frac{(n-1)\pi}{n}\right)$
& $\cdots$ & $2\cos\left(\frac{ (n-1)^2 \pi}{n}\right)$\\
\hline
\end{tabular}
\end{center}
When $n$ is even we have:
\begin{center}
\begin{tabular}{|c|c|c|c|c|c|c|}
\hline 
 & $1$ & $sr^{2i}$ & $sr^{2i+1}$ & $r^{\pm 1}$ & $\cdots$ & $r^{n/2}$\\
\hline 
\hline 
$\chi_0$ & $1$ & $1$ & $1$ & $1$ & $\cdots$ &  $1$\\
\hline 
$\chi_1$ & $1$ & $-1$ & $-1$ & $1$ & $\cdots$ & $1$\\
\hline 
$\chi_2$ & $1$ & $-1$ & $1$ & $-1$ & $\cdots$ &
$(-1)^{n/2}$\\
\hline 
$\chi_3$ & $1$ & $1$ & $-1$ & $-1$ & $\cdots$ & $(-1)^{n/2}$\\
\hline 
$\sigma_{1}$ & $2$ & $0$ & $0$ & $2\cos\left(\frac{2\pi}{n}\right)$ & $\cdots$
 & $-2$\\
\hline 
$\vdots$ & $\vdots$ & $\vdots$ & $\vdots$ & $\vdots$ & $\ddots$ & $\vdots$\\
\hline
$\sigma_{n/2-1}$ & $2$ & $0$ & $0$ &
$2\cos\left(\frac{(n-2)\pi}{n}\right)$ & $\cdots$ & $-2$\\
\hline
\end{tabular}
\end{center}
We can check that these are all irreducible by checking that their norms
are $1$ with respect to the inner product.  By summing the squares of
the values on $1$, we see that we have all of them. 
\end{example}

\begin{example}
The character table for the alternating group on $4$ letters
is:
\begin{center}
\begin{tabular}{|c|c|c|c|c|}
\hline 
 & $()$ & $(1\ 2)(3\ 4)$ & $(1\ 2\ 3)$ & $(1\ 3\ 2)$\\
\hline 
\hline 
$\chi_1$ & $1$ & $1$ & $1$ & $1$\\
\hline 
$\chi_2$ & $1$ & $1$ & $\zeta$ & $\zeta^{-1}$\\
\hline 
$\chi_3$ & $1$ & $1$ & $\zeta^{-1}$ & $\zeta$\\
\hline 
$\chi_4$ & $3$ & $-1$ & $0$ & $0$\\
\hline
\end{tabular}
\end{center}
where $\zeta$ is a primitive cube root of unity.
\end{example}

\begin{example}
The character table for the alternating group on $5$ letters
is:
\begin{center}
\begin{tabular}{|c|c|c|c|c|c|}
\hline 
 & $()$ & $(1\ 2)(3\ 4)$ & $(1\ 2\ 3)$ & $(1\ 2\ 3\ 4\ 5)$ & $(1\ 3\ 5\ 2\ 4)$\\
\hline 
\hline 
$\chi_1$ & $1$ & $1$ & $1$ & $1$ & $1$\\
\hline 
$\chi_2$ & $3$ & $-1$ & $0$ & $\varphi$ & $-1/\varphi$\\
\hline 
$\chi_3$ & $3$ & $-1$ & $0$ & $-1/\varphi$ & $\varphi$\\
\hline 
$\chi_4$ & $4$ & $0$ & $1$ & $-1$ & $-1$\\
\hline 
$\chi_5$ & $5$ & $1$ & $-1$ & $0$ & $0$\\
\hline
\end{tabular}
\end{center}
where $\varphi$ is the golden ratio $\frac{1}{2}(1+\sqrt{5})$.
\end{example}

%%%%%%%%%%%%%%%%%%
%%%%%%%%%%%%%%%%%%

\bibliographystyle{alpha}
\bibliography{rep_theory}

\end{document}


