\documentclass[12pt]{article}

\usepackage{amssymb,amsmath,amsthm}

% theorem styles (I like everything to have the same counter)
\theoremstyle{plain}
\newtheorem{theorem}{Theorem}[section]
\newtheorem{lemma}[theorem]{Lemma}
\newtheorem{proposition}[theorem]{Proposition}
\newtheorem{corollary}[theorem]{Corollary}
\newtheorem{conjecture}[theorem]{Conjecture}
\newtheorem{exercise}[theorem]{Exercise}
\theoremstyle{definition}
\newtheorem{definition}[theorem]{Definition}
\newtheorem{example}[theorem]{Example}
\theoremstyle{remark}
\newtheorem{remark}[theorem]{Remark}

% some numbering settings
\numberwithin{equation}{section}

\usepackage{fancyhdr}
\usepackage{lastpage}
\setlength{\headheight}{15.2pt}
\renewcommand{\footrulewidth}{0.4pt}% default is 0pt
\setlength{\footskip}{30pt}
\pagestyle{fancy}

\makeatletter
\let\ps@plain\ps@fancy 
\makeatother

\lhead{MATH 742}
\chead{Character Theory}
\rhead{Spring 2023}
\lfoot{Last Revised: \today}
\cfoot{}
\rfoot{\thepage\ of \pageref{LastPage} }

\begin{document}

%%%%%%%%%%%%%%%%%%
%%%%%%%%%%%%%%%%%%
\title{Character Theory}
\author{Alexander Duncan}

\maketitle

Some of the content here can be found in
\S 14--15 of \cite{AlperinBell} and
\S 2 of \cite{Serre}.
TODO

Throughout this document, we make the standing assumptions that $G$
is a finite group and $k$ is a field of characteristic coprime to the
order of $G$.
This allows us to appeal to Maschke's theorem:
every finite-dimensional representation is completely reducible.
Thus, every indecomposable representation is also irreducible.

\section{Representation Ring}

Let $R^+_k(G)$ be the set of isomorphism classes of
finite-dimensional representations of $G$ over $k$.
Given representations $V$ and $W$, let $[V]$ and $[W]$ denote their
images in $R^+_k(G)$.
We define addition on $R^+_k(G)$ via
\[ [V] + [W] = [V \oplus W] \]
and multiplication on $R^+_k(G)$ via
\[ [V] \cdot [W] = [V \otimes W]. \]
We write $0$ for $[0]$ and $1$ for $[k]$.
(One should check that these operations are well-defined.)

For representations $U,V,W$, let $u=[U]$, $v=[V]$, and $w=[W]$.
Write $0 = [0]$ for the zero representation and $1=[k]$
for the trivial representation.
Now, standard isomorphisms give rise to properties of $R^+_k(G)$ as follows:
\begin{align*}
U \oplus V &\cong V \oplus U & u+v&=v+u\\
U \oplus (V \oplus W) &\cong (U \oplus V) \oplus W & u+(v+w)&=(u+v)+w\\
0 \oplus V &\cong V & 0+v&=v\\
V \oplus 0 &\cong V & v+0&=v\\
U \otimes V &\cong V \otimes U & uv&=vu\\
U \otimes (V \otimes W) &\cong (U \otimes V) \otimes W & u(vw)&=(uv)w\\
k \otimes V &\cong V & 1v&=v\\
V \otimes k &\cong V & v1&=v\\
0 \otimes V &\cong 0 & 0v&=0\\
V \otimes 0 &\cong 0 & v0&=0\\
U \otimes (V \oplus W) &\cong (U \otimes V) \oplus (U \otimes W) &
u(v+w)&=uv+uw\\
(V \oplus W) \otimes U &\cong (V \otimes U) \oplus (V \otimes U) &
(v+w)u&=vu+wu
\end{align*}

We would like to conclude that $R^+_k(G)$ is a ring,
but we are missing additive inverses.  Indeed, no non-zero
representation has an additive inverse for dimensional reasons.
However, $R^+_k(G)$ does have the structure of a commutative \emph{rig}
(``a riNg without Negatives'').
In particular, $R^+_k(G)$ is a commutative monoid under addition,
and the non-zero elements of $R^+_k(G)$ form a commutative monoid under
multiplication.

Let $S_k(G)$ be the set of isomorphism classes of
irreducible representations of $G$ over $k$.
From the Krull-Schmidt theorem, every element $[V]$ of $R^+_G(k)$
can be written uniquely as a finite linear combination
\[
[V] = m_1 [W_1] + \cdots + m_r [W_r]
\]
where $[W_1],\ldots,[W_r] \in S_k(G)$ and $m_1,\ldots, m_r$ are
non-negative integers.
Thus there is an additive monoid isomorphism
$R^+_G(k) \cong \mathbb{N}^{\oplus S_k(G)}$ .
(We will see that $S_k(G)$ is always finite, but we distinguish between
the direct sum rather and the direct product until we prove this).

However, while the additive structure on $R^+_G(k)$ is very nice,
the \emph{multiplicative} structure is considerably more subtle.
Let $X$ be an indexing set for $S_k(G) = \{[W_i]\}_{i \in X}$.
The multiplicative structure is completely determined by the
non-negative integers $c^{ij}_\ell$ in the expressions
\[
[W_i] \cdot [W_j] = \sum_{\ell \in X} c^{ij}_\ell [W_\ell]
\]
where $i,j$ vary independently over all values of $X$.

\begin{remark}
The numbers $c^{ij}_\ell$ are called the \emph{Clebsch-Gordan numbers}
for the group $G$.  After selecting bases for the various $W_i$,
we have explicit matrices defining the isomorphism between
$W_i \otimes W_j$ and the decomposition into irreducible representations;
the entries of this matrix are called the \emph{Clebsch-Gordan
coefficients}.  The case where $G$ is the (infinite) special orthogonal
group $SO(3)$ is of special interest in quantum physics and quantum
chemistry.
\end{remark}

\subsection{Virtual Representations}

A \emph{virtual representation} $X$ is an element of the free abelian
group on $S_k(G)$.
More concretely, there is an expression
\[
X = m_1 [W_1] + m_2 [W_2] + \cdots + m_r [W_r]
\]
where $[W_1],\ldots,[W_r]$ are distinct isomorphism classes of
irreducible representations
and $m_1,\ldots, m_r$ are integers.
If all the elements $m_1,\ldots,m_r$ are non-negative,
then the virtual representation is an (isomorphism class of) a
representation via the evident interpretation in $R^+_k(G)$.

\begin{definition}
The \emph{representation ring $R_k(G)$ of $G$ over $k$} is
the additive group of virtual representations with multiplication
obtained by extending the multiplication of $R^+_k(G)$ by linearity.
\end{definition}

\begin{remark}
There are way more theorems about \emph{rings} than \emph{rigs}.
Thus, we ``adjoined negatives'' to make our rig $R^+_k(G)$
into a ring $R_k(G)$.
This procedure is an additive analog to forming the ring of fractions of a ring.
In general, one can similarly construct the \emph{Grothendieck group}
or \emph{group completion} $G$ from a commutative monoid $M$.
If the monoid $M$ is a rig, then the group $G$ becomes a ring.
However, we only get a nice embedding $M \hookrightarrow G$ if
$M$ is a \emph{cancellative} monoid.
\end{remark}


%%%%%%%%%%%%%%%%%%
%%%%%%%%%%%%%%%%%%

\bibliographystyle{alpha}
\bibliography{rep_theory}

\end{document}


