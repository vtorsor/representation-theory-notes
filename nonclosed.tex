\documentclass[12pt]{article}

\usepackage{amssymb,amsmath,amsthm}

% theorem styles (I like everything to have the same counter)
\theoremstyle{plain}
\newtheorem{theorem}{Theorem}[section]
\newtheorem{lemma}[theorem]{Lemma}
\newtheorem{proposition}[theorem]{Proposition}
\newtheorem{corollary}[theorem]{Corollary}
\newtheorem{conjecture}[theorem]{Conjecture}
\newtheorem{exercise}[theorem]{Exercise}
\theoremstyle{definition}
\newtheorem{definition}[theorem]{Definition}
\newtheorem{example}[theorem]{Example}
\theoremstyle{remark}
\newtheorem{remark}[theorem]{Remark}

% some numbering settings
\numberwithin{equation}{section}

\usepackage{fancyhdr}
\usepackage{lastpage}
\setlength{\headheight}{15.2pt}
\renewcommand{\footrulewidth}{0.4pt}% default is 0pt
\setlength{\footskip}{30pt}
\pagestyle{fancy}

\makeatletter
\let\ps@plain\ps@fancy 
\makeatother

\lhead{MATH 742}
\chead{Representations over non-closed fields}
\rhead{Spring 2023}
\lfoot{Last Revised: \today}
\cfoot{}
\rfoot{\thepage\ of \pageref{LastPage} }

\begin{document}

%%%%%%%%%%%%%%%%%%
%%%%%%%%%%%%%%%%%%
\title{Representations over non-closed fields}
\author{Alexander Duncan}

\maketitle

Throughout, $k$ is a field of characteristic $0$
and $\overline{k}$ is the algebraic closure of $k$.
Let $\Gamma_k$ be the absolute Galois group of $k$.
(For those uncomfortable with infinite Galois extensions,
one can instead replace $\overline{k}$ with the Galois closure
of the finitely many relevant finite field extensions
showing up in any of the contexts we discuss below.)

Much of this is inspired from \cite[\S{12--13}]{Serre}.

\section{Representation Ring via Modules}

Let $G$ be a finite group.

Recall that $kG$ is the group algebra of $G$ over $k$.
Since $k$ has characterstic $0$, we know $kG$ is semisimple.
We have a canonical decomposition
\[
kG \cong \oplus_{i=1}^r A_i
\]
where $A_1,\ldots,A_r$ are simple $k$-algebras.

Each $A_i$ is a central simple $F_i$-algebra
for a finite field extension $F_i/k$.
Let $n_i = \deg(A_i)$ be the \emph{degree} of $A_i$.
Equivalently, we have $n_i^2 = \dim_{F_i}(A_i)$
or
\[
A_i \otimes_{F_i} \overline{k} \cong
\operatorname{M}_{n_i}(\overline{k})
\]
as $\overline{k}$-algebras.

More precisely, there exists a central division $F_i$-algebra $D_i$
and a positive integer $m_i$ such that
\[
A_i \cong \operatorname{M}_{m_i}(D_i)
\]
as $F_i$-algebras.
Both $m_i$ and the isomorphism class of $D_i$ are uniquely determined.
Let $d_i := \operatorname{ind}(A_i)$ denote the Schur index of $A_i$.
We have $d_i = \deg(D_i)$, which is equivalent to
$d_i^2=\dim_{F_i}(D_i)$.  Also, we have $n_i=m_id_i$.

Each $A_i$ has a simple module $V_i$, which is unique up to isomorphism.
Since $V_i \cong D_i^{\oplus n_i}$, we see that
\[
\dim_k V_i = [F_i:k]m_id_i^2 = [F_i:k]d_in_i .
\]
The vector spaces $V_1,\ldots,V_r$ are precisely the
irreducible representations of $G$ over $k$.
The \emph{Schur index} of $V_i$ is defined to be the
Schur index $d_i$ of the associated algebra $A_i$.

Let $X_i :=
\operatorname{Hom}_{k-\operatorname{alg}}(F_i,\overline{k})$.
Observe that $X_i$ has a left action of $\Gamma_k$ by
post-composition.
Informally, $X_i$ can be thought of as the set of roots of the minimal
polynomial of a primitive element of $F_i/k$.
We have a $\overline{k}$-algebra isomorphism
\[
\omega : F_i \otimes_k \overline{k} \cong
\bigoplus_{\sigma \in X_i} \overline{k}
\]
via $\omega(f \otimes \ell)_{\sigma} := \sigma(f)\ell$.

For each $\sigma \in X_i$, we may define a tensor product
$A_i \otimes_\sigma \overline{k}$ of $F_i$-algebras
where we use the structure map $F_i \to A_i$ on the left and
the homomorphism $\sigma : F_i \to \overline{k}$ on the right.
Via the sequence of isomorphisms of $\overline{k}$-algebras
\[
A_i \otimes_k \overline{k} \cong 
A_i \otimes_{F_i} F_i \otimes_k \overline{k}
\cong A_i \otimes_{F_i} \bigoplus_{\sigma \in X_i} \overline{k}
\]
we obtain an isomorphism
\[
\Omega : A_i \otimes_k \overline{k} \cong
\bigoplus_{\sigma \in X_i} A_i \otimes_\sigma \overline{k} .
\]

Let $W_{i,\sigma}$ be the irreducible representation
of $\overline{k}G$ corresponding to $A_i \otimes_\sigma \overline{k}$.
We conclude that we have the following decomposition
\[
V_i \otimes_k \overline{k} \cong
\bigoplus_{\sigma \in X_i} W_{i,\sigma}^{\oplus d_i}
\]
for each irreducible representation $V_i$ of $kG$.

If $\rho_{i,\sigma} : G \to \operatorname{GL}(W_{i,\sigma})$ are
the corresponding maps, then notice that
\[
\tau \circ \rho_{i,\sigma} = \rho_{i,\tau(\sigma)}
\]
for every $\tau \in \Gamma$.
Thus, there is a left action of $\Gamma$ on the set of irreducible
representations of $G$ over $\overline{k}$
with orbits corresponding to the $\Gamma$-sets $X_i$.

\section{Character Theory}

Here we will assume that $\overline{k}$ (and therefore $k$)
is a specific subfield of $\mathbb{C}$.
This is essentially harmless in view of the Lefschetz principle.
For the suspicious, assume that $\mathbb{C}$ is just bizarre notation for
$\overline{k}$ throughout the remainder of the document.

Recall that the representation ring $R_k(G)$ of $G$ over $k$
is the additive group of virtual representations over $k$
with multiplication obtained extending the tensor product
on the subrig $R_k^+(G)$.
From above, we see that $R_k(G)$ is a free abelian group
on the set $[V_1],\ldots,[V_r]$.

The map $[V] \mapsto [V \otimes_k \overline{k}]$
gives a canonical injective ring homomorphism
\[
R_k(G) \hookrightarrow R_{\mathbb{C}} = R(G)
\]
and therefore a canonical injective ring homomorphism
\[
R_k(G) \hookrightarrow C(G)
\]
where $C(G)$ is the set of complex class functions on $G$.

Just as in the complex case, given a representation $(V,\rho)$
of $G$ over $k$, we define the character
$\chi_V : G \to k$
via the trace $\chi_V(g) := \operatorname{tr}(\rho(g))$.
This agrees with the map $R_k(G) \hookrightarrow C(G)$ defined above.
Thus, we identify $R_k(G)$ with a subring of $R(G)$ and $C(G)$ in what
follows.

Let $\chi_1,\ldots,\chi_r$ be the irreducible representations of $G$
over $k$ corresponding to the algebras $A_1,\ldots, A_r$ above.
Let $\psi_{i,\sigma} : G \to \overline{k}=\mathbb{C}$ be the character
corresponding to the irreducible representation $W_{i,\sigma}$
defined in the previous section.

\begin{theorem}
For every $1 \le i \le r$, we have
\[
\chi_i = d_i \sum_{\sigma \in X_i} \psi_{i,\sigma} .
\]
In particular, the characters $\chi_1,\ldots, \chi_r$
are an orthogonal basis for the subspace $R_k(G)$ of $C(G)$.
\end{theorem}

\begin{proof}
The description of $\chi_i$ follows from the results in the previous
section.
The orthogonality follows from the fact that the $\psi_{i,\sigma}$
are orthogonal.
\end{proof}

Note that we only have the $\{\chi_1,\ldots,\chi_r\}$ is an
\emph{orthogonal} basis; it is not necessarily orthonormal.

There are in fact \emph{three} different traces that one may consider
for each irreducible representation $V_i$.
We have the ordinary trace $\chi_i(g) \in k$ as an element of
$\operatorname{End}_k(V_i)$.
We may also define $\phi_i(g) \in F_i$ as the trace of $g$ in
$\operatorname{End}_{F_i}(V_i)$.
Finally, we have $\psi_i(g) \in F_i$ as the reduced trace
$\operatorname{Trd}(g)$ viewing $g$ as an element of the central simple
$F_i$-algebra $A_i$.
These traces are all related via
\[
\chi_i = \operatorname{Tr}_{F_i/k} \circ \phi_i, \quad
\phi_i = d_i \psi_i
\]
where we recall that $d_i$ is the Schur index of the central simple
$F_i$-algebra $A_i$ (equivalently, the representation $V_i$).
Finally for each $\sigma \in X_i$, we see that $\psi_{i,\sigma} = \sigma
\circ \psi_i$.
This gives another way of obtaining the decomposition of $V_i \otimes_k
\overline{k}$ above.

\subsection{Definability of Representations}

Recall, from the previous section, that the absolute Galois group
permutes the irreducible representations of $\overline{k}G$.
Indeed, if $\rho_{i,\sigma} : G \to \operatorname{GL}(W_{i,\sigma})$
is the map then we have
\[
\tau \circ \rho_{i,\sigma} = \rho_{i,\tau(\sigma)}
\]
for $\tau \in \Gamma$.
This action is identical to the one obtained by acting on the
characters
\[
\tau \circ \phi_{i,\sigma} = \phi_{i,\tau(\sigma)}
\]
where we recall $\phi_{i,\sigma}: G \to \overline{k}$ is just a
particular class function.

\begin{theorem}
Let $\overline{R}_k(G)$ be the subset of $R(G)$ consisting of characters
whose class functions whose image is defined over $k$.
Then $\overline{R}_k(G)$ has basis
\[
\frac{\chi_1}{d_1},
\frac{\chi_2}{d_2},
\cdots,
\frac{\chi_r}{d_r}.
\]
\end{theorem}

Thus $\overline{R}_k(G)$ can be completely recovered from the character
table over $\mathbb{C}$.  Namely, one can reconstruct the $\chi_i/d_i$
as the sums of $\Gamma$-orbits of the complex characters.
To reconstruct $\overline{R}_k(G)$, one needs to know the Schur indices.

\begin{proposition}
If $G$ is an abelian group,
then $R_k(G) = \overline{R}_k(G)$
\end{proposition}

\begin{proof}
If $G$ is abelian, then $kG$ is commutative, so all constituent
division algebras are fields.  Thus all Schur indices are trivial.
\end{proof}

We are now in a position to prove an important theorem of Brauer, which
essentially reduces the study of definability of representations of
arbitrary fields of characteristic $0$ to the case of cyclotomic fields.

\begin{theorem}
If $G$ is a finite group of exponent $e$,
then every representation of $G$ over an arbitrary field of
characteristic $0$
is conjugate to a representation defined over the
cyclotomic field $\mathbb{Q}(\zeta_e)$.
\end{theorem}

\begin{proof}
Let $K=\mathbb{Q}(\zeta_e)$.  We want to show that $R_K(G)=R(G)$.
By Brauer's theorem, we may write a general element $\chi$ of $R(G)$
in the form
\[
\rho = \sum_{i=1}^m c_i\operatorname{Ind}_{H_i}^G(\tau_i)
\]
where the $c_i$'s are integers, the $H_i$ are subgroups of $G$,
and each $\tau_i$ is a one-dimensional representations of $R(H_i)$.
A one-dimensional representation of $H_i$ is defined over
$\mathbb{Q}(\zeta_e)$ since $H_i$ has exponent at most $e$.
We conclude that $\chi$ is an integral linear combination of
representations induced from $R_K(H_i)$, so is in $R_K(G)$ as desired.
\end{proof}

\bibliographystyle{alpha}
\bibliography{rep_theory}

\end{document}


