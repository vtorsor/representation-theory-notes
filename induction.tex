\documentclass[12pt]{article}

\usepackage{amssymb,amsmath,amsthm}

% theorem styles (I like everything to have the same counter)
\theoremstyle{plain}
\newtheorem{theorem}{Theorem}[section]
\newtheorem{lemma}[theorem]{Lemma}
\newtheorem{proposition}[theorem]{Proposition}
\newtheorem{corollary}[theorem]{Corollary}
\newtheorem{conjecture}[theorem]{Conjecture}
\newtheorem{exercise}[theorem]{Exercise}
\theoremstyle{definition}
\newtheorem{definition}[theorem]{Definition}
\newtheorem{example}[theorem]{Example}
\theoremstyle{remark}
\newtheorem{remark}[theorem]{Remark}

% some numbering settings
\numberwithin{equation}{section}

\usepackage{fancyhdr}
\usepackage{lastpage}
\setlength{\headheight}{15.2pt}
\renewcommand{\footrulewidth}{0.4pt}% default is 0pt
\setlength{\footskip}{30pt}
\pagestyle{fancy}

\makeatletter
\let\ps@plain\ps@fancy 
\makeatother

\lhead{MATH 742}
\chead{Induced Representations}
\rhead{Spring 2023}
\lfoot{Last Revised: \today}
\cfoot{}
\rfoot{\thepage\ of \pageref{LastPage} }

\begin{document}

%%%%%%%%%%%%%%%%%%
%%%%%%%%%%%%%%%%%%
\title{Induced Representations}
\author{Alexander Duncan}

\maketitle

{\bf
Throughout this document, we make the standing assumptions that $G$
is a finite group and $k$ is a field of characteristic coprime to the
order of $G$.}

``The basics'' of induced representations can be found in essentially
every introductory character theory text.  For example:
\cite[\S{16}]{AlperinBell},
\cite[\S{5.8-5.11}]{Etingof},
\cite[\S{3.3}]{FultonHarris},
\cite[\S{XVIII.6--9}]{Lang}, or
\cite[\S{3.3}]{Serre}.
However, we go a bit further then ``the basics'' here;
see \cite[\S{7--10}]{Serre}.

%%%%%%%%%%%%%%%%%%
%%%%%%%%%%%%%%%%%%

\section{Group Actions}

Let $G$ be a group.

\begin{definition}
A (left) \emph{$G$-set} is a set $X$ with a (left) action of a finite
group $G$.  Given $G$-sets $X$ and $Y$, a \emph{morphism of $G$-sets}
or \emph{$G$-equivariant map} is a function $f : X \to Y$
such that $f(gx)=gf(x)$ for all $x \in X$ and $g \in G$.
An \emph{isomorphism of $G$-sets} is a $G$-equivariant bijection. 
\end{definition}

Let $H$ be a subgroup $H$ of $G$.  Recall that a \emph{left coset} is a
subset of $G$ of the form
\[
gH = \{ gh \mid h \in H \}
\]
for some $g \in G$.  Recall that the left cosets partition $G$.
We denote by $G/H$ the set of left cosets of $H$ in $G$.
We do \emph{not} expect $G/H$ to have a group structure unless $H$ is
normal.

\begin{proposition}
The set $G/H$ has a canonical transitive left $G$-action.
\end{proposition}

\begin{proof}
The action is defined via
\[
g \cdot (aH) := (ga)H
\]
for $g,a \in G$.
Suppose $aH=bH$ for $a,b \in G$.  Then $a=bh$ for some $h \in H$.
Thus $ga=gbh$ for all $g \in G$.  Thus $gaH=gbH$.
We conclude that the action is well-defined.
Checking the remaining two axioms for a group action are left as an exercise.
\end{proof}

\begin{remark}
The definition of the left action on $G/H$ is more subtle than the proof
suggests.  We do \emph{not} have an analogous right action on $G/H$
since it is not well-defined.  Asking that the right action be
well-defined is equivalent to asking that $H$ be a normal subgroup.
\end{remark}

\begin{exercise}
Let $H,K$ be subgroups of $G$.
The $G$-sets $G/H$ and $G/K$ are isomorphic if and only if
$H$ and $K$ are conjugate subgroups of $G$.
\end{exercise}

Recall that a $G$-set $X$ is \emph{transitive} if $X$ is nonempty
and, for all $x \in X$ and
$y \in Y$, there exists a $g \in G$ such that $gx=y$.

\begin{exercise}
Every $G$-set can be written uniquely as a disjoint union of
transitive $G$-sets.
\end{exercise}

\begin{proposition}
Every non-empty transitive $G$-set $X$ is isomorphic to $G/H$
for some subgroup $H \subseteq G$.
\end{proposition}

\begin{proof}
Since $X$ is non-empty, we may select some $x \in X$.
Let $H$ be the stabilizer of $x$ in $G$:
\[
H = \operatorname{Stab}_G(x) = \{ g \in G \mid gx=x \}.
\]
We claim there is a unique $G$-equivariant function
$f : X \to G/H$ such that $f(x)=H$.
Indeed, since $G$ acts transitively on $X$,
the condition $f(gx)=gf(x)$ determines the value of $f$ at every point
of $X$.  To see that this is well-defined, we simply observe that
$f(gx)=f(g'x)$ is equivalent to $g^{-1}g'x=x$, which is equivalent to
$g^{-1}g' \in H$.
\end{proof}

\subsection{Right actions}

Everything stated above for \emph{left} group actions has a natural
generalization to \emph{right} group actions, which we mostly leave as
an exercise for the reader.
We use the notation $H \backslash G$ for the set of right cosets of $H$
in $G$.

The following exercises are strongly recommended if you haven't
though about the distinction of left and right actions before:

\begin{exercise}
If $X$ is a \emph{left} $G$-set and $G$ is abelian, then there is a
\emph{right} action on $X$ via $x \cdot g := gx$ for $g \in G$ and $x \in X$.
\end{exercise}

\begin{exercise}
If $X$ is a \emph{left} $G$-set, then we obtain a \emph{right}
action on $X$ via $x \cdot g := g^{-1}x$ for $g \in G$ and $x \in X$.
\end{exercise}

\begin{exercise}
The function $f : G/H \to H\backslash G$ given by $f(gH)=Hg$ is a
well-defined bijection if and only if $H$ is normal in $G$.
\end{exercise}

\subsection{Double cosets}

Let $H, K$ be subgroups of $G$.

\begin{definition}
A \emph{(H,K)-double coset of G} is a set of the form
\[
HgK := \{ hgk \mid h \in H, k \in K \}
\]
for some $g \in G$.
The set of double cosets of $G$ is denoted
$H \backslash G / K$.
\end{definition}

There are several different equivalent perspectives on double cosets.
Double cosets are the equivalence classes for the equivalence relation
$g \sim hgk$ for $h \in H$ and $k \in K$.
Double cosets are the orbits of the left $H$-action on $G/K$.
Double cosets are the orbits of the right $K$-action on $H\backslash G$.
Finally, double cosets are the orbits of the \emph{left} action
of $H \times K$ on $G$ via $(h,k)\cdot g := hgk^{-1}$.

Each of these perspectives is useful in different contexts.
An immediate consequence is that double cosets partition $G$;
in other words, $HgK$ and $Hg'K$ are either disjoint or equal
and every element of $G$ is contained in some double coset.
Another consequence is that each double coset is both a union of
right cosets of $H$ in $G$ and a union of left cosets of $K$ in $G$.

Note that unlike ordinary cosets, double cosets are not guaranteed to
all have the same size.  However, their size can be easily computed
from the orbit stabilizer theorem:

\begin{exercise}
If $G$ is finite, then
\[
|HgK| = \frac{|H| |K|}{|H \cap g K g^{-1}|}=
\frac{|H| |K|}{|g^{-1}Hg \cap K|}.
\]
\end{exercise}

\section{Introduction}

\subsection{Permutation Representations}

\subsection{Monomial Representations}

\section{Example of Non-Induced Representation}

The following may seem rather technical at the moment, but it will be
relevant later in the course.  We introduce it now so that we have a
non-abelian example of a irreducible representation that is
\emph{not} an induced representation.

Recall that \emph{Hamilton's quaternion algebra} $\mathbb{H}$
is the division $\mathbb{R}$-algebra
with basis $\{1,i,j,k\}$ and multiplication determined by
$ij=k$, $jk=i$, $ki=j$ and $i^2=j^2=k^2=-1$.
We have an embedding
$\mathbb{H} \hookrightarrow \operatorname{M}_2(\mathbb{C})$
by extending
\[
1 \mapsto \begin{pmatrix} 1&0\\0&1 \end{pmatrix} \quad
i \mapsto \begin{pmatrix} i&0\\0&-i \end{pmatrix} \quad
j \mapsto \begin{pmatrix} 0&1\\-1&0 \end{pmatrix} \quad
k \mapsto \begin{pmatrix} 0&i\\i&0 \end{pmatrix}
\]
by linearity.
This embedding is a homomorphism of $\mathbb{R}$-algebras.
(Note that there is no embedding of $\mathbb{H}$ into
the \emph{real} $2\times 2$ matrices.)

A general element $x$ of $\mathbb{H}$ of the form
\[
x = a + bi + cj + dk
\]
where $a,b,c,d \in \mathbb{R}$.
We define the \emph{conjugate} of $x$ as
\[
\overline{x} = a -bi-cj-dk,
\]
and the \emph{norm} of $x$ as
\[
\|x\| = \sqrt{\overline{x} x} = \sqrt{a^2+b^2+c^2+d^2},
\]
which is a non-negative real number.
We see that $\mathbb{H}$ is a division algebra
since $x^{-1} = \overline{x}/\|x\|$ is an explicit formula for the
inverse when $x \ne 0$.

A general element $x$ of $\mathbb{H}$
is a \emph{Lipschitz quaternion} if
$a,b,c,d \in \mathbb{Z}$.
The element $x$ is a \emph{Hurwitz quaternion}
if either $a,b,c,d$ are all integers, or all half-integers.
The set $L$ of Lipschitz quaternions and the set $H$ of Hurwitz
quaternions are both subrings of Hamilton's quaternions.
The unit groups for these rings can be found by observing that the norm of
an invertible element must be $1$.

The \emph{quaternionic group} $Q_8 = \{\pm 1, \pm i, \pm j, \pm k\}$ is
the unit group $U(L)$ of the ring $L$.
It is evident from the embedding
$\mathbb{H} \to \operatorname{M}_2(\mathbb{C})$ described above
that $Q_8$ has a representation by ``twisted permutation
matrices.''
Namely, there is a choice of basis where the group acts by permuting
the basis elements and multiplying them by scalars.

The unit group $U(H)$ of the ring $H$ contains $Q_8$ and the 16 elements
\[
\frac{ \pm 1 \pm i \pm j \pm k }{2}
\]
where all possible sign combinations are permitted.
One checks that $\frac{1}{2}(1+i+j+k)$ has order $6$, and thus we have a
representation with image
\[
\left\langle \begin{pmatrix} i&0\\0&-i \end{pmatrix},
\begin{pmatrix} 0&1\\-1&0 \end{pmatrix},
\frac{1}{2} \begin{pmatrix} 1+i&1+i\\-1+i&1-i \end{pmatrix}\right\rangle
.
\]
This is a representation that \emph{cannot} be written wholly with
``twisted permutation matrices'' in any basis.

The group $U(H)$ is sometimes called the ``binary tetrahedral group''
due to the following description.

\begin{exercise}
Let $Z$ be the center of $U(H)$.
Prove that the center $Z$ of $U(H)$ has order $2$ and
$U(H)/Z $ is isomorphic to $A_4$.
\end{exercise}

\begin{exercise}
Prove that $U(H)$ is isomorphic to $Q_8 \rtimes (\mathbb{Z}/3\mathbb{Z})$.
\end{exercise}

\begin{exercise}
Prove that $U(H)$ is isomorphic to the special linear group
$\operatorname{SL}_2(\mathbb{F}_3)$ over the field of $3$ elements.
\end{exercise}


\bibliographystyle{alpha}
\bibliography{rep_theory}

\end{document}


