\documentclass[12pt]{article}

\usepackage{amssymb,amsmath,amsthm}

% theorem styles (I like everything to have the same counter)
\theoremstyle{plain}
\newtheorem{theorem}{Theorem}[section]
\newtheorem{lemma}[theorem]{Lemma}
\newtheorem{proposition}[theorem]{Proposition}
\newtheorem{corollary}[theorem]{Corollary}
\newtheorem{conjecture}[theorem]{Conjecture}
\newtheorem{exercise}[theorem]{Exercise}
\theoremstyle{definition}
\newtheorem{definition}[theorem]{Definition}
\newtheorem{example}[theorem]{Example}
\theoremstyle{remark}
\newtheorem{remark}[theorem]{Remark}

% some numbering settings
\numberwithin{equation}{section}

\usepackage{fancyhdr}
\usepackage{lastpage}
\setlength{\headheight}{15.2pt}
\renewcommand{\footrulewidth}{0.4pt}% default is 0pt
\setlength{\footskip}{30pt}
\pagestyle{fancy}

\makeatletter
\let\ps@plain\ps@fancy 
\makeatother

\lhead{MATH 742}
\chead{Induced Representations}
\rhead{Spring 2023}
\lfoot{Last Revised: \today}
\cfoot{}
\rfoot{\thepage\ of \pageref{LastPage} }

\begin{document}

%%%%%%%%%%%%%%%%%%
%%%%%%%%%%%%%%%%%%
\title{Induced Representations}
\author{Alexander Duncan}

\maketitle

``The basics'' of induced representations can be found in essentially
every introductory character theory text.  For example:
\cite[\S{16}]{AlperinBell},
\cite[\S{5.8-5.11}]{Etingof},
\cite[\S{3.3}]{FultonHarris},
\cite[\S{XVIII.6--9}]{Lang}, or
\cite[\S{3.3}]{Serre}.
However, we go a bit further then ``the basics'' here;
see \cite[\S{7--10}]{Serre}.

%%%%%%%%%%%%%%%%%%
%%%%%%%%%%%%%%%%%%

\section{Group Actions}

Let $G$ be a group.

\begin{definition}
A (left) \emph{$G$-set} is a set $X$ with a (left) action of a finite
group $G$.  Given $G$-sets $X$ and $Y$, a \emph{morphism of $G$-sets}
or \emph{$G$-equivariant map} is a function $f : X \to Y$
such that $f(gx)=gf(x)$ for all $x \in X$ and $g \in G$.
An \emph{isomorphism of $G$-sets} is a $G$-equivariant bijection. 
\end{definition}

Let $H$ be a subgroup of $G$.  Recall that a \emph{left coset of $H$ in
$G$} is a subset of $G$ of the form
\[
gH = \{ gh \mid h \in H \}
\]
for some $g \in G$.  Recall that the left cosets partition $G$.
We denote by $G/H$ the set of left cosets of $H$ in $G$.
We do \emph{not} expect $G/H$ to have a group structure unless $H$ is
normal.

\begin{proposition}
The set $G/H$ has a canonical transitive left $G$-action.
\end{proposition}

\begin{proof}
The action is defined via
\[
g \cdot (aH) := (ga)H
\]
for $g,a \in G$.
Suppose $aH=bH$ for $a,b \in G$.  Then $a=bh$ for some $h \in H$.
Thus $ga=gbh$ for all $g \in G$.  Thus $gaH=gbH$.
We conclude that the action is well-defined.
Checking the remaining two axioms for a group action are left as an exercise.
\end{proof}

\begin{remark}
The definition of the left action on $G/H$ is more subtle than a
superficial reading of the proof would suggest.
We do \emph{not} have an analogous right action on $G/H$
since it is not well-defined.  Asking that the right action be
well-defined is equivalent to asking that $H$ be a normal subgroup.
\end{remark}

\begin{exercise}
Let $H,K$ be subgroups of $G$.
The $G$-sets $G/H$ and $G/K$ are isomorphic if and only if
$H$ and $K$ are conjugate subgroups of $G$.
\end{exercise}

Recall that a $G$-set $X$ is \emph{transitive} if $X$ is nonempty
and, for all $x \in X$ and
$y \in Y$, there exists a $g \in G$ such that $gx=y$.

\begin{exercise}
Every $G$-set can be written uniquely as a disjoint union of
transitive $G$-sets.
\end{exercise}

\begin{proposition}
Every non-empty transitive $G$-set $X$ is isomorphic to $G/H$
for some subgroup $H \subseteq G$.
\end{proposition}

\begin{proof}
Since $X$ is non-empty, we may select some $x \in X$.
Let $H$ be the stabilizer of $x$ in $G$:
\[
H = \operatorname{Stab}_G(x) = \{ g \in G \mid gx=x \}.
\]
We claim there is a unique $G$-equivariant function
$f : X \to G/H$ such that $f(x)=H$.
Indeed, since $G$ acts transitively on $X$,
the condition $f(gx)=gf(x)$ determines the value of $f$ at every point
of $X$.  To see that this is well-defined, we simply observe that
$f(gx)=f(g'x)$ is equivalent to $g^{-1}g'x=x$, which is equivalent to
$g^{-1}g' \in H$.
\end{proof}

\subsection{Right actions}

Everything stated above for \emph{left} group actions has a natural
generalization to \emph{right} group actions, which we mostly leave as
an exercise for the reader.
We use the notation $H \backslash G$ for the set of right cosets of $H$
in $G$.

The following exercises are strongly recommended if you haven't
thought about the distinction of left and right actions before:

\begin{exercise}
If $X$ is a \emph{left} $G$-set and $G$ is abelian, then there is a
\emph{right} action on $X$ via $x \cdot g := gx$ for $g \in G$ and $x \in X$.
\end{exercise}

\begin{exercise}
If $X$ is a \emph{left} $G$-set, then we obtain a \emph{right}
action on $X$ via $x \cdot g := g^{-1}x$ for $g \in G$ and $x \in X$.
\end{exercise}

\begin{exercise}
The function $f : G/H \to H\backslash G$ given by $f(gH)=Hg$ is a
well-defined bijection if and only if $H$ is normal in $G$.
\end{exercise}

\subsection{Double cosets}

Let $H, K$ be subgroups of $G$.

\begin{definition}
A \emph{(H,K)-double coset of G} is a set of the form
\[
HgK := \{ hgk \mid h \in H, k \in K \}
\]
for some $g \in G$.
The set of double cosets of $G$ is denoted
$H \backslash G / K$.
\end{definition}

There are several different equivalent perspectives on double cosets.
Double cosets are the equivalence classes for the equivalence relation
where $g \sim hgk$ for all $h \in H$ and $k \in K$.
Double cosets are the orbits of the left $H$-action on $G/K$.
Double cosets are the orbits of the right $K$-action on $H\backslash G$.
Finally, double cosets are the orbits of the \emph{left} action
of $H \times K$ on $G$ via $(h,k)\cdot g := hgk^{-1}$.

Each of these perspectives is useful in different contexts.
An immediate consequence is that double cosets partition $G$;
in other words, $HgK$ and $Hg'K$ are either disjoint or equal
and every element of $G$ is contained in some double coset.
Another consequence is that each double coset is both a union of
right cosets of $H$ in $G$ and a union of left cosets of $K$ in $G$.

Note that unlike ordinary cosets, double cosets are not guaranteed to
all have the same size.  However, their size can be easily computed
from the orbit stabilizer theorem:

\begin{exercise}
If $G$ is finite, then
\[
|HgK| = \frac{|H| |K|}{|H \cap g K g^{-1}|}=
\frac{|H| |K|}{|g^{-1}Hg \cap K|}.
\]
\end{exercise}

\section{Definition of Induced Representation}

Let $H$ be a subgroup of a finite group $G$.

Given a finite dimensional
representation $(V,\rho)$ of $G$, we define the \emph{restriction to
$H$} as the representation $(V,\rho|_H)$ where the underlying vector
space is the same, but the group homomorphism $\rho|_H : H \to
\operatorname{GL}(V)$ has domain $H$.
We often write $\operatorname{Res}_H^G(V)$ or $V {\downarrow}^G_H$
to emphasize the $H$-action, even though the vector space is the same.

Induced representations are a method of going in the other direction.
We begin with a representation of $H$ and produce a representation of
$G$.
It is convenient to define induced representations via their universal
property:

\begin{definition}
Suppose $H$ is a subgroup of a finite group $G$ and
$W$ is a finite dimensional representation of $H$. 
The \emph{induced representation} is a representation
$\operatorname{Ind}_H^G W$ of $G$ together with an
$H$-equivariant map $\iota : W \to \operatorname{Ind}_H^G W$ such that
for any other representation $V$ of $G$ and $H$-equivariant map
$\varphi : W \to V$, there exists a unique $G$-equivariant map
$\psi: \operatorname{Ind}_H^G W \to V$ such that
$\psi \circ \iota = \varphi$.
\end{definition}

An alternative notation for $\operatorname{Ind}_H^G W$ is $W
{\uparrow}_H^G$.

As usual, the universal property is economical to state and is useful
theoretically, but it is less easy to see that it exists.
On the other hand, an explicit construction may have various
unimportant ``implementation details'' that vary from author to author
and are not of fundamental importance.
Nevertheless, a direct construction must be carried out and we
will need them later.

\begin{proposition} \label{prop:ind_construction}
Induced representations exist and are unique up to unique isomorphism.
\end{proposition}

\begin{proof}
The uniqueness statement follows from the universal mapping property by
abstract nonsense.  The interesting part is existence.

We are given a finite group $G$ with subgroup $H$ and a representation
$(W,\sigma)$ of $H$.
Choose a system of distinct representatives $X$ for $G/H$.
Let $V$ be the direct sum $\bigoplus_{x \in X} W$.
Labeling the $x$th summand of $V$ by $xW$,
we obtain isomorphisms $\phi_x : W \to xW$.

We now observe that
\[
\operatorname{End}_k(V) = \bigoplus_{x \in X} \bigoplus_{y \in X}
\operatorname{Hom}_k(xW,yW)
\]
has a ``block matrix'' structure.
Given an element $\psi \in \operatorname{End}_k(V)$,
denote by $\psi^{y,x}$ the restriction to each
$\operatorname{Hom}_k(xW,yW)$.

Fix $g \in G$; we will define a map
$\rho_g \in \operatorname{End}_k(V)$ by defining each component
$\rho_g^{y,x}$.
For all $g \in G$ and $x,y \in X$, we define
\[
\rho_g^{y,x} =
\begin{cases}
\phi_y \circ \sigma_{y^{-1}gx} \circ \phi_x^{-1} & \textrm{if $gxH=yH$,}\\
0 & \textrm{otherwise},
\end{cases}
\]
where we observe that $y^{-1}gx \in H$ so the evaluation of $\sigma$
makes sense.

It is now tedious, but straightforward, to check that $(V,\rho)$ is linear
representation of $G$ satisfying the desired universal property.
\end{proof}

\begin{remark}
Choose a system of distinct representatives $X$ for $G/H$ and
write $\operatorname{Ind}^G_H W = \bigoplus_{x \in X} W$
as in the proof of Proposition~\ref{prop:ind_construction}.
There is an $H$-equivariant linear map
\[
\pi : \operatorname{Ind}^G_H W \to W
\]
obtained by taking the sum
\[
\pi\left( (w_x)_{x \in X} \right) :=
\sum_{x \in X} w_x
\]
over all $x$ in a system of distinct representatives $X$.
The map $\pi$ turns out not to depend on the choice of $X$.
and can be thought of as a generalization of the trace.

Defining this map gives a different universal property:
given any representation $V$ of $G$ along with an $H$-equivariant map
$\varphi : V \to W$, there exists a unique $G$-equivariant
map $\psi : V \to \operatorname{Ind}^G_H W$ such that
$\pi \circ \psi = \varphi$.
In a more general context, objects satisfying the two universal
properties may not coincide and we refer to those satisfying this second
construction as \emph{coinduced representations.}
\end{remark}

\section{Monomial Representations}

In practice, many representations are induced from a subgroups.  We've
already seen many examples.  In particular:

\begin{proposition}
If $V$ is a permutation representation of a finite group $G$,
then $V$ is a direct sum of representations induced from the trivial
representation of subgroup.
\end{proposition}

\begin{proof}
Recall that $V$ has a basis $\{e_x\}_{x \in X}$ where $X$ is a $G$-set.
Since all $G$-sets are a disjoint union of transitive $G$-sets,
we see that $V$ is a direct sum of permutation representations
corresponding to transitive $G$-sets.  Thus, without loss of generality
we may assume $X \cong G/H$ for some subgroup $H \subseteq G$.

Choosing a system of distinct representatives $S$ for $G/H$,
we now identify the direct summands $sW$ in the proof of
Proposition~\ref{prop:ind_construction} with the spans of
the basis elements $\{ e_{sH} \}$ in $V$.
One checks that the group action of $G$ is identical in both cases.
\end{proof}

\begin{example}
The regular representation of $G$ is precisely the
representation $\operatorname{Ind}^G_1 k$
obtained by induction from the trivial representation of the trivial
group.
\end{example}

\begin{definition}
A representation $(V,\rho)$ of a group $G$ is a \emph{monomial
representation} or a \emph{twisted permutation representation} if 
there exists a basis $\{e_1,\ldots,e_n\}$ of $V$
such that, for all $1 \le i \le n$ and $g \in G$
we have
\[ \rho_g(e_i)=\lambda e_j\]
for some $1 \le j \le n$ and $\lambda \in k^\times$.
\end{definition}

Monomial representations are those with a basis such that the
corresponding matrices have
exactly one non-zero element in each row and column.
Permutation representations are those where that non-zero element is
always equal to one.

\begin{exercise}
If $V$ is a monomial representation of a finite group $G$,
then $V$ is a direct sum of representations induced from one-dimensional
representations of subgroups.
\end{exercise}

\section{Characters of Induced Representations}

In this section, we assume that the base field $k$ is $\mathbb{C}$.

\begin{definition}
Given a class function $f : G \to \mathbb{C}$, define
$\operatorname{Res}_H^G f : H \to \mathbb{C}$ as the
class function obtained by restricting the domain to $H$.
\end{definition}

It is clear that the character of $\operatorname{Res}_H^G V$
is $\operatorname{Res}_H^G \chi_V$.

\begin{definition}
Given a class function $f : H \to \mathbb{C}$, we construct a new class
function $\operatorname{Ind}_H^G f : G \to \mathbb{C}$ by the formula
\[
(\operatorname{Ind}_H^G f)(g) =
\frac{1}{|H|}\sum_{\substack{a \in G\\a^{-1}ga \in H}} f(a^{-1}ga)
\]
for $g \in G$.
\end{definition}

\begin{proposition}
The character of $\operatorname{Ind}_H^G V$
is $\operatorname{Ind}_H^G \chi_V$.
\end{proposition}

\begin{proof}
We use the block matrix notation from
Proposition~\ref{prop:ind_construction}
and set $X$ to be a system of distinct representatives for $G/H$.
We want to compute the trace of $\rho_g$ for some $g \in G$.
This is equivalent to summing the traces of the elements
$\rho_g^{x,x}$ for all $x \in X$.
We observe that $\rho_g^{x,x} = 0$ unless $gx=xh$ for some
$h \in H$.
Thus we are only summing over $x \in X$ such that
$x^{-1}gx \in H$.  In this case, we have
$\rho_g^{x,x}=\phi_x \circ \sigma_h \circ \phi_x^{-1}$,
which has the same trace as $\sigma_h$ where $h=x^{-1}gx$.
Thus, we have the formula
\[
(\operatorname{Ind}_H^G f)(g) =
\sum_{\substack{x \in X\\x^{-1}gx \in H}} f(x^{-1}gx) .
\]
Since $f$ is a class function for $H$,
we see that $f(x^{-1}gx)=f(y^{-1}gy)$ whenever $xH=yH$.
Thus, summing over $G$ instead of $X$ amounts to
multiplication by the size of the cosets $|H|$.
\end{proof}

\begin{exercise}
Prove that the function
$\operatorname{Res}_H^G : C(G) \to C(H)$
is a ring homomorphism.
\end{exercise}

\begin{exercise}
Prove that the function
$\operatorname{Ind}_H^G : C(H) \to C(G)$
is an additive group homomorphism,
but a ring homomorphism only if $G=H$.
(Hint: what is $(\operatorname{Ind}_H^G 1)(1)$?)
\end{exercise}

\begin{proposition}
If $\varphi$ is a class function of $H$ and $\psi$ is a class function
of $G$, then
$\operatorname{Ind}^G_H ( (\operatorname{Res^G_H} \psi) \cdot\varphi)
= \psi \cdot (\operatorname{Ind}^G_H \varphi)$.
\end{proposition}

\begin{proof}
Since $\psi(a^{-1}ga)=\psi(g)$ for all $a \in G$, we have
\[
\frac{1}{|H|}\sum_{\substack{a \in G\\a^{-1}ga \in H}} \psi(a^{-1}ga)\varphi(a^{-1}ga)
 =
 \psi(g) \left(\frac{1}{|H|}\sum_{\substack{a \in G\\a^{-1}ga \in H}} \varphi(a^{-1}ga)
\right)
\]
for all $g \in G$.
\end{proof}

\begin{corollary}
The image of $\operatorname{Ind}^G_H$ is an ideal of $C(G)$.
\end{corollary}

The above facts are stated for the ring of class functions
$C(G)$, but one can check that they also hold for the representation
ring $R(G)$ as well.

\section{Frobenius Reciprocity}

\begin{theorem}
Suppose $V$ is a representation of $G$
and $W$ is a representation of a subgroup $H$ of $G$.
There is a canonical isomorphism
\[
\operatorname{Hom}^G_k(\operatorname{Ind}_H^G W, V) =
\operatorname{Hom}^H_k(W,\operatorname{Res}_H^G  V).
\]
\end{theorem}

\begin{proof}
Recall that we have an $H$-equivariant map
$\iota : W \to \operatorname{Ind}_H^G(W)$ by the definition of the
induced representation.
Thus, given an element
$\psi \in \operatorname{Hom}^G_k(\operatorname{Ind}_H^G W, V)$,
we obtain an element
$\varphi \in \operatorname{Hom}^H_k(W,\operatorname{Res}_H^G  V)$
by the composition $\varphi = \psi \circ \iota$.
We uniquely recover $\psi$ from $\varphi$ from the universal mapping
property, showing the two sets are in bijection.
We've seen that composition of linear maps induces a linear
transformation on Hom-spaces so we have the desired isomorphism
of vector spaces.
\end{proof}

An immediate corollary is the famous:

\begin{theorem}[Frobenius Reciprocity]
If $\varphi$ is a class function of $H$ and $\psi$ is a class function
of $G$, then
\[
(\operatorname{Ind}_H^G \varphi, \psi)_G
=
(\varphi, \operatorname{Res}_H^G \psi)_H
\]
where we use subscripts to emphasize that one inner product is in $C(G)$
while the other is in $C(H)$.
\end{theorem}

\begin{proof}
In the case where $\varphi$ and $\psi$ are characters,
we have
\begin{align*}
&(\operatorname{Ind}_H^G \varphi, \psi)_G\\
=& \dim_\mathbb{C}\ \operatorname{Hom}^G_\mathbb{C}(\operatorname{Ind}_H^G W, V)\\
=& \dim_\mathbb{C}\ \operatorname{Hom}^H_\mathbb{C}(W,\operatorname{Res}_H^G  V)\\
=&(\varphi, \operatorname{Res}_H^G \psi)_H .
\end{align*}
This extends to all class functions by linearity since the characters
are a spanning set.
\end{proof}

\section{Mackey decomposition}

Here we learn how restriction and induction interact.

\begin{theorem}[Mackey Decomposition]
Let $K,H$ be subgroups of $G$ and suppose $(W,\sigma)$ is a representation of
$H$.
Then
\[
\operatorname{Res}^G_K \operatorname{Ind}^G_H W
\cong
\bigoplus_{[x] \in K\backslash G / H}
\operatorname{Ind}_{H_x}^K W_x
\]
where $H_x = xHx^{-1} \cap K$, and $(W_x,\sigma_x)$ is a representation
of $H_x$ with underlying vector space $W$
defined by $\sigma_x(g)=\sigma(x^{-1}gx)$ for all $g \in H_x$.
\end{theorem}

\begin{proof}
Let $X$ be a system of distinct representatives for $G/H$.
Observe that $H_x$ is the stabilizer of the $K$-action on each $x \in X$.
Let $X_1,\ldots,X_r$ be the subsets of $X$ corresponding to the
$K$-orbits on $G/H$.
Observe that the sets $\{X_i\}$ correspond bijectively to the
double cosets in $K \backslash G / H$.

We recall the explicit description of $\tau = \operatorname{Ind}^G_H \rho$
from Proposition~\ref{prop:ind_construction}
as a direct sum $\bigoplus_{x \in X} xW$,
with isomorphisms $\phi_x: W \to xW$ and $\tau_g$ decomposing as
\[
\rho_g^{y,x} =
\begin{cases}
\phi_y \circ \rho_{y^{-1}gx} \circ \phi_x^{-1} & \textrm{if $gxH=yH$,}\\
0 & \textrm{otherwise},
\end{cases}
\]
in each $\operatorname{Hom}_k(xW,yW)$
for $x,y \in X$ and $g \in G$.

Let $W_i$ be the sum of the subspaces $xW$ for $x \in X_i$.
We have
\[
\operatorname{Ind}^G_H W = \bigoplus_{i=1}^r W_i
\]
where each $W_i$ is easily seen to be $K$-stable.
It remains to prove that
$W_i \cong \operatorname{Ind}_{H_x}^K W_x$ for some choice of $x \in X_i$.

Again by Proposition~\ref{prop:ind_construction},
we have $\tau_i = \operatorname{Ind}_{H_x}^K \sigma_x$
with $\psi_y : W_x \to yW_x$ and
\[
\rho_g^{z,y} =
\begin{cases}
\psi_z \circ \rho_{z^{-1}gy} \circ \psi_y^{-1} & \textrm{if $gyH_x=zH_x$,}\\
0 & \textrm{otherwise},
\end{cases}
\]
in each $\operatorname{Hom}_k(yW,zW)$
for $y,z \in X_i$ and $g \in K$.

For each $y \in X_i$, define $\pi_y : yW \to yW_x$
via $\pi_y = \psi_y \circ \phi_y^{-1}$.
The direct sum $\bigoplus_{y \in X_i} \pi_y$
is an isomorphism from $W_i$ to $\operatorname{Ind}_{H_x}^K \sigma_x$.
One checks that this is $K$-equivariant as desired.
\end{proof}

We now turn to an application where $H=K$.
As in the statement of the Mackey Decomposition Theorem,
for any element $x \in G$, we define $H_x = xHx^{-1} \cap H$.
Observe that while $\operatorname{Res}^H_{H_x} W$
and $W_x$ are both representations of $H_x$,
they are not isomorphic in general!
If $\chi$ is the character for $W$ in $C(H)$, let $\chi_x$ denote the
character of $W_x$ in $C(H_x)$.

\begin{theorem}[Mackey Irreducibility Criterion]
The representation $\operatorname{Ind}^G_H W$ is irreducible
if and only if $W$ is irreducible and,
for every $x \in G \setminus H$,
we have $( \chi_x, \operatorname{Res}^H_{H_x} \chi )=0$.
\end{theorem}

\begin{proof}
We use Frobenius Reciprocity and the Mackey Decomposition liberally:
\begin{align*}
&( \operatorname{Ind}^G_H \chi, \operatorname{Ind}^G_H \chi )\\
=&( \chi, \operatorname{Res}^G_H \operatorname{Ind}^G_H \chi )\\
=&( \chi, \sum_{[x] \in H\backslash G / H}
\operatorname{Ind}_{H_x}^H \chi_x )\\
=&\sum_{[x] \in H\backslash G / H} ( \chi, 
\operatorname{Ind}_{H_x}^H \chi_x )\\
=&\sum_{[x] \in H\backslash G / H} ( \operatorname{Res}_{H_x}^H\chi, 
 \chi_x ).
\end{align*}
If $x \in H$, then they are equal and
$\chi=\operatorname{Res}_{H_x}^H\chi = \chi_x$.
Thus
\[
( \operatorname{Ind}^G_H \chi, \operatorname{Ind}^G_H \chi )
=(\chi,\chi)+
\sum_{\substack{[x] \in H\backslash G / H\\x \not\in  H}}
( \operatorname{Res}_{H_x}^H\chi,  \chi_x )
\]
Since $\operatorname{Res}_{H_x}^H\chi$ and $\chi_x$ are actual
characters of $H_x$, their inner product is always a non-negative integer.
Moreover, $(\chi,\chi)$ is a strictly positive integer.
Thus
$( \operatorname{Ind}^G_H \chi, \operatorname{Ind}^G_H \chi )=1$
if and only if $(\chi,\chi)=1$ and every other
summand vanishes.
The theorem follows.
\end{proof}

\begin{corollary}
Suppose $H$ is normal in $G$.
The representation $\operatorname{Ind}^G_H \rho$ is irreducible
if and only if $\rho$ is irreducible and $\rho_x \ne \rho$
for all $x \not\in H$.
\end{corollary}

\begin{proof}
When $H$ is normal, $H_x=H$ since $xHx^{-1}=H$ for all $x \in G$.
In particular, $\operatorname{Res}^G_H(\rho)=\rho$.
Assuming $\chi$ is irreducible, the condition
$\left(\chi_x,\operatorname{Res}^G_H(\chi)\right)=0$
is equivalent to $\chi_x \ne \chi$ and the result follows from
the Mackey Irreducibility Criterion.
\end{proof}


\section{Examples}

\subsection{Dihedral Groups}

Let $G$ be the dihedral group
\[
D_{2n} = \langle r,s \mid s^2, r^n, (sr)^2 \rangle
\]
and let $H$ be the cyclic subgroup $\langle r \rangle$.
Let $(W,\sigma)$ be the $1$-dimensional representation of $H$
given by $\sigma(r)=(\zeta)$ where $\zeta$ is an $n$th root of unity.

We will construct $\operatorname{Ind}^G_H W$, which we call $(V,\rho)$
for brevity.
We use the notation from Proposition~\ref{prop:ind_construction}.
Let $X = \{e, s \}$ be a set of distinct representatives for $G/H$.
To determine $\rho$, we require understanding $\rho_g^{x,y}$
for generators $g=r,s$ and all pairs $x,y \in X$.
After computing
\[
r(eH)=eH,\ r(sH)=sH,\ s(eH)=sH,\ r(eH)=rH 
\]
we find
\[
\rho(r)=
\begin{pmatrix} \sigma(e^{-1}re) & 0 \\ 0 & \sigma(s^{-1}rs) \end{pmatrix}
=\begin{pmatrix} \zeta & 0 \\ 0 & \zeta^{-1} \end{pmatrix}
\]
and
\[
\rho(s)=
\begin{pmatrix} 0 & \sigma(e^{-1}ss) \\ \sigma(s^{-1}se) & 0 \end{pmatrix}
=\begin{pmatrix} 0 & 1 \\ 1 & 0 \end{pmatrix}.
\]

\subsection{The Symmetric Group on $4$ Letters}

Let $G$ be the symmetric group $S_4$ on $4$ letters
and let $H$ be the subgroup generated by $(1\ 2)(3\ 4)$
and $(2\ 3)$.
Observe that $H$ is a non-normal subgroup of index $3$,
which is isomorphic to $D_8$.
Let $(W,\sigma)$ be the $1$-dimensional representation of $H$
given by $\sigma((1\ 2)(3\ 4))=(-1)$ and $\sigma((2\ 3))=(1)$.
Note that the kernel of $\sigma$ is generated by $(1\ 4)$ and $(2\ 3)$.

We construct the induced representation $(V,\rho)$ using
Proposition~\ref{prop:ind_construction} as before.
First, let us take $X=\{ e, (1\ 2\ 3), (1\ 3\ 2) \}$ as our system of
distinct representatives for $G/H$.
We compute
\begin{align*}
(1\ 2)(3\ 4)&\mapsto
\begin{pmatrix}
-1 & 0 & 0\\
0 & 1 & 0\\
0 & 0 & -1\\
\end{pmatrix}\\
(2\ 3)&\mapsto
\begin{pmatrix}
1 & 0 & 0\\
0 & 0 & 1\\
0 & 1 & 0\\
\end{pmatrix}\\
(1\ 2\ 3)&\mapsto
\begin{pmatrix}
0 & 0 & 1\\
1 & 0 & 0\\
0 & 1 & 0\\
\end{pmatrix}
\end{align*}
in this case.

\subsection{A Non-Induced Representation}

Here we develop a non-abelian example of a irreducible representation
that is \emph{not} an induced representation.
I recommend only skimming this section on a first reading.

Recall that \emph{Hamilton's quaternion algebra} $\mathbb{H}$
is the division $\mathbb{R}$-algebra
with basis $\{1,i,j,k\}$ and multiplication determined by
$ij=k$, $jk=i$, $ki=j$ and $i^2=j^2=k^2=-1$.
We have an embedding
$\mathbb{H} \hookrightarrow \operatorname{M}_2(\mathbb{C})$
by extending
\[
1 \mapsto \begin{pmatrix} 1&0\\0&1 \end{pmatrix} \quad
i \mapsto \begin{pmatrix} i&0\\0&-i \end{pmatrix} \quad
j \mapsto \begin{pmatrix} 0&1\\-1&0 \end{pmatrix} \quad
k \mapsto \begin{pmatrix} 0&i\\i&0 \end{pmatrix}
\]
by linearity.
This embedding is a homomorphism of $\mathbb{R}$-algebras.
(Note that there is no embedding of $\mathbb{H}$ into
the \emph{real} $2\times 2$ matrices.)

A general element $x$ of $\mathbb{H}$ of the form
\[
x = a + bi + cj + dk
\]
where $a,b,c,d \in \mathbb{R}$.
We define the \emph{conjugate} of $x$ as
\[
\overline{x} = a -bi-cj-dk,
\]
and the \emph{norm} of $x$ as
\[
\|x\| = \sqrt{\overline{x} x} = \sqrt{a^2+b^2+c^2+d^2},
\]
which is a non-negative real number.
We see that $\mathbb{H}$ is a division algebra
since $x^{-1} = \overline{x}/\|x\|$ is an explicit formula for the
inverse when $x \ne 0$.

A general element $x$ of $\mathbb{H}$
is a \emph{Lipschitz quaternion} if
$a,b,c,d \in \mathbb{Z}$.
The element $x$ is a \emph{Hurwitz quaternion}
if either $a,b,c,d$ are all integers, or all half-integers.
The set $L$ of Lipschitz quaternions and the set $H$ of Hurwitz
quaternions are both subrings of Hamilton's quaternions.
The unit groups for these rings can be found by observing that the norm of
an invertible element must be $1$.

The \emph{quaternionic group} $Q_8 = \{\pm 1, \pm i, \pm j, \pm k\}$ is
the unit group $U(L)$ of the ring $L$.
It is evident from the embedding
$\mathbb{H} \to \operatorname{M}_2(\mathbb{C})$ described above
that $Q_8$ has a representation by ``twisted permutation
matrices.''
Namely, there is a choice of basis where the group acts by permuting
the basis elements and multiplying them by scalars.

The unit group $U(H)$ of the ring $H$ contains $Q_8$ and the 16 elements
\[
\frac{ \pm 1 \pm i \pm j \pm k }{2}
\]
where all possible sign combinations are permitted.
One checks that $\frac{1}{2}(1+i+j+k)$ has order $6$, and thus we have a
representation with image
\[
\left\langle \begin{pmatrix} i&0\\0&-i \end{pmatrix},
\begin{pmatrix} 0&1\\-1&0 \end{pmatrix},
\frac{1}{2} \begin{pmatrix} 1+i&1+i\\-1+i&1-i \end{pmatrix}\right\rangle
.
\]

\begin{exercise}
Prove that $U(H)$ does not have a subgroup of index $2$.
\end{exercise}

An immediate consequence of this exercise is that $U(H)$ cannot be
induced from a $1$-dimensional representation.

Recall that $A_4$ is often called the ``tetrahedral group''
since it is isomorphic to the set of orthogonal matrices preserving a
regular tetrahedron.
Consequently, the group $U(H)$ is sometimes called the ``binary tetrahedral group''
due to the following exercise.

\begin{exercise}
Let $Z$ be the center of $U(H)$.
Prove that the center $Z$ of $U(H)$ has order $2$ and
$U(H)/Z $ is isomorphic to $A_4$.
\end{exercise}


\begin{exercise}
Prove that $U(H)$ is isomorphic to $Q_8 \rtimes (\mathbb{Z}/3\mathbb{Z})$.
\end{exercise}

\begin{exercise}
Prove that $U(H)$ is isomorphic to the special linear group
$\operatorname{SL}_2(\mathbb{F}_3)$ over the field of $3$ elements.
\end{exercise}

\section{Applications}

\section{Artin and Brauer's Theorem}



\bibliographystyle{alpha}
\bibliography{rep_theory}

\end{document}


