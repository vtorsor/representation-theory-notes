\documentclass[12pt]{article}

\usepackage{amssymb,amsmath,amsthm}

% theorem styles (I like everything to have the same counter)
\theoremstyle{plain}
\newtheorem{theorem}{Theorem}[section]
\newtheorem{lemma}[theorem]{Lemma}
\newtheorem{proposition}[theorem]{Proposition}
\newtheorem{corollary}[theorem]{Corollary}
\newtheorem{conjecture}[theorem]{Conjecture}
\newtheorem{exercise}[theorem]{Exercise}
\theoremstyle{definition}
\newtheorem{definition}[theorem]{Definition}
\newtheorem{example}[theorem]{Example}
\theoremstyle{remark}
\newtheorem{remark}[theorem]{Remark}

% some numbering settings
\numberwithin{equation}{section}

\usepackage{fancyhdr}
\usepackage{lastpage}
\setlength{\headheight}{15.2pt}
\renewcommand{\footrulewidth}{0.4pt}% default is 0pt
\setlength{\footskip}{30pt}
\pagestyle{fancy}

\makeatletter
\let\ps@plain\ps@fancy 
\makeatother

\lhead{MATH 742}
\chead{Induced Representations}
\rhead{Spring 2023}
\lfoot{Last Revised: \today}
\cfoot{}
\rfoot{\thepage\ of \pageref{LastPage} }

\begin{document}

%%%%%%%%%%%%%%%%%%
%%%%%%%%%%%%%%%%%%
\title{Induced Representations}
\author{Alexander Duncan}

\maketitle

{\bf
Throughout this document, we make the standing assumptions that $G$
is a finite group and $k$ is a field of characteristic coprime to the
order of $G$.}

``The basics'' of induced representations can be found in essentially
every introductory character theory text.  For example:
\cite[\S{16}]{AlperinBell},
\cite[\S{5.8-5.11}]{Etingof},
\cite[\S{3.3}]{FultonHarris},
\cite[\S{XVIII.6--9}]{Lang}, or
\cite[\S{3.3}]{Serre}.
However, we go a bit further then ``the basics'' here;
see \cite[\S{7--10}]{Serre}.

%%%%%%%%%%%%%%%%%%
%%%%%%%%%%%%%%%%%%

\section{Induced Representations}

\subsection{Non-example}

The following may seem rather technical at the moment, but it will be
relevant later in the course.  We introduce it now so that we have a
non-abelian example of a irreducible representation that is
\emph{not} an induced representation.

Recall that \emph{Hamilton's quaternion algebra} $\mathbb{H}$
is the division $\mathbb{R}$-algebra
with basis $\{1,i,j,k\}$ and multiplication determined by
$ij=k$, $jk=i$, $ki=j$ and $i^2=j^2=k^2=-1$.
We have an embedding
$\mathbb{H} \hookrightarrow \operatorname{M}_2(\mathbb{C})$
by extending
\[
1 \mapsto \begin{pmatrix} 1&0\\0&1 \end{pmatrix} \quad
i \mapsto \begin{pmatrix} i&0\\0&-i \end{pmatrix} \quad
j \mapsto \begin{pmatrix} 0&1\\-1&0 \end{pmatrix} \quad
k \mapsto \begin{pmatrix} 0&i\\i&0 \end{pmatrix} \quad
\]
by linearity.
This embedding is a homomorphism of $\mathbb{R}$-algebras.
(Note that there is no embedding of $\mathbb{H}$ into
the \emph{real} $2\times 2$ matrices.)

A general element $x$ of $\mathbb{H}$ of the form
\[
x = a + bi + cj + dk
\]
where $a,b,c,d \in \mathbb{R}$.
We define the \emph{conjugate} of $x$ as
\[
\overline{x} = a -bi-cj-dk,
\]
and the \emph{norm} of $x$ as
\[
\|x\| = \sqrt{\overline{x} x} = \sqrt{a^2+b^2+c^2+d^2},
\]
which is a non-negative real number.
We see that $\mathbb{H}$ is a division algebra
since $x^{-1} = \overline{x}/\|x\|$ is an explicit formula for the
inverse when $x \ne 0$.

A general element $x$ of $\mathbb{H}$
is a \emph{Lipschitz quaternion} if
$a,b,c,d \in \mathbb{Z}$.
The element $x$ is a \emph{Hurwitz quaternion}
if either $a,b,c,d$ are all integers, or all half-integers.
The set $L$ of Lipschitz quaternions and the set $H$ of Hurwitz
quaternions are both subrings of Hamilton's quaternions.
The unit groups for these rings can be found by observing that the norm of
an invertible element must be $1$.

The \emph{quaternionic group} $Q_8 = \{\pm 1, \pm i, \pm j, \pm k\}$ is
the unit group $U(L)$ of the ring $L$.
It is evident from the embedding
$\mathbb{H} \to \operatorname{M}_2(\mathbb{C})$ described above
that $Q_8$ has a representation by ``twisted permutation
matrices.''
Namely, there is a choice of basis where the group acts by permuting
the basis elements and multiplying them by scalars.

The unit group $U(H)$ of the ring $H$ contains $Q_8$ and the 16 elements
\[
\frac{ \pm 1 \pm i \pm j \pm k }{2}
\]
where all possible sign combinations are permitted.
One checks that $\frac{1}{2}(1+i+j+k)$ has order $6$, and thus we have a
representation with image
\[
\left\langle \begin{pmatrix} i&0\\0&-i \end{pmatrix},
\begin{pmatrix} 0&1\\-1&0 \end{pmatrix},
\frac{1}{2} \begin{pmatrix} 1+i&1+i\\-1+i&1-i \end{pmatrix}\right\rangle
.
\]
This is a representation that \emph{cannot} be written wholly with
``twisted permutation matrices'' in any basis.

The group $U(H)$ is sometimes called the ``binary tetrahedral group''
due to the following description.

\begin{exercise}
Let $Z$ be the center of $U(H)$.
Prove that the center $Z$ of $U(H)$ has order $2$ and
$U(H)/Z $ is isomorphic to $A_4$.
\end{exercise}

\begin{exercise}
Prove that $U(H)$ is isomorphic to $Q_8 \rtimes (\mathbb{Z}/3\mathbb{Z})$.
\end{exercise}

\begin{exercise}
Prove that $U(H)$ is isomorphic to the special linear group
$\operatorname{SL}_2(\mathbb{F}_3)$ over the field of $3$ elements.
\end{exercise}


\bibliographystyle{alpha}
\bibliography{rep_theory}

\end{document}


