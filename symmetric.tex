\documentclass[12pt]{article}

\usepackage{amssymb,amsmath,amsthm}
\usepackage{ytableau}

% theorem styles (I like everything to have the same counter)
\theoremstyle{plain}
\newtheorem{theorem}{Theorem}[section]
\newtheorem{lemma}[theorem]{Lemma}
\newtheorem{proposition}[theorem]{Proposition}
\newtheorem{corollary}[theorem]{Corollary}
\newtheorem{conjecture}[theorem]{Conjecture}
\newtheorem{exercise}[theorem]{Exercise}
\theoremstyle{definition}
\newtheorem{definition}[theorem]{Definition}
\newtheorem{example}[theorem]{Example}
\theoremstyle{remark}
\newtheorem{remark}[theorem]{Remark}

% some numbering settings
\numberwithin{equation}{section}

\usepackage{fancyhdr}
\usepackage{lastpage}
\setlength{\headheight}{15.2pt}
\renewcommand{\footrulewidth}{0.4pt}% default is 0pt
\setlength{\footskip}{30pt}
\pagestyle{fancy}

\makeatletter
\let\ps@plain\ps@fancy 
\makeatother

\lhead{MATH 742}
\chead{Induced Representations}
\rhead{Spring 2023}
\lfoot{Last Revised: \today}
\cfoot{}
\rfoot{\thepage\ of \pageref{LastPage} }

\begin{document}

%%%%%%%%%%%%%%%%%%
%%%%%%%%%%%%%%%%%%
\title{Symmetric Groups and General Linear Groups}
\author{Alexander Duncan}

\maketitle

Much of this material is drawn from
\cite[\S{4,6,A}]{FultonHarris},
\cite[\S{5.12--5.19}]{Etingof},
\cite[\S{I}]{Macdonald}, and
\cite[\S{7}]{Stanley2}.

%%%%%%%%%%%%%%%%%%
%%%%%%%%%%%%%%%%%%

\section{Partitions}

Recall that a \emph{partition} $\lambda$ is a sequence of decreasing
non-negative integers
$\lambda_1 \ge \lambda_2 \ge \cdots$ that is eventually $0$.
Some standard terminology:
\begin{itemize}
\item The non-zero $\lambda_i$ are the \emph{parts} of $\lambda$.
\item The number $\ell(\lambda)$ of parts is the \emph{length} of $\lambda$.
\item The sum $|\lambda|=\sum_{i \ge 0} \lambda_i$ is the \emph{weight} of $\lambda$.
\item A ``partition of $n$'' is a partition of weight $n$.
\item $\lambda \vdash n$ means $\lambda$ is a partition of $n$.
\item The number $m_i(\lambda)$ of parts equal to $i$ is
the \emph{multiplicity} of $i$ in $\lambda$.
\item Partitions can be written $\lambda_1 + \lambda_2 + \cdots + \lambda_r$.
\item We may use the shorthand $\lambda = (1^{m_1} 2^{m_2} \cdots r^{m_r})$.
\end{itemize}

Partitions are often drawn as \emph{Young tableaux}.  This is just a
series of empty boxes where each row contains $\lambda_i$ boxes.
There are some differing conventions between different areas of math, so
read any source carefully.

Given a partition $\lambda$, the \emph{conjugate partition}
is the partition $\lambda'$ obtained by reflecting the Young diagram in
the downwards-right diagonal line.  More explicitly,
$\lambda'_i$ is the number of parts such that $\lambda_j \ge i$. 

\begin{example}
The partitions of $4$ are as follows:
\begin{center}
\begin{tabular}{ccccc}
4 & 3+1 & 2+2 & 2+1+1 & 1+1+1+1 \\
\ydiagram{4} & \ydiagram{3,1} & \ydiagram{2,2} & \ydiagram{2,1,1} &
\ydiagram{1,1,1,1}
\end{tabular}
\end{center}
\end{example}

\begin{example}
Let $\lambda = (4,4,2,2,2,1)$.
We may also write $\lambda$ as $4+4+2+2+2+1$ or $1^12^34^2$.
We have weights $\lambda_1=4$, $\lambda_2=4$,
$\lambda_3=2$,
$\lambda_4=2$,
$\lambda_5=2$, and
$\lambda_6=1$.
We have length $\ell(\lambda)=6$, weight $|\lambda|=15$,
and multiplicities $m_1(\lambda)=1$, $m_2(\lambda)=3$,
$m_3(\lambda)=0$, and $m_4(\lambda)=2$.
The Young tableau is
\begin{center}
\ydiagram{4,4,2,2,2,1}.
\end{center}
The conjugate partition $\lambda'$ is $(6,5,2,2)$.
\end{example}

We use the notation $\operatorname{Par}(n)$ to denote the set of
partitions of $n$.  And the notation $\operatorname{Par}$ for the set of
all partitions of all weights.

\bibliographystyle{alpha}
\bibliography{rep_theory}

\end{document}


