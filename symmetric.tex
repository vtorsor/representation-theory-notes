\documentclass[12pt]{article}

\usepackage{amssymb,amsmath,amsthm}
\usepackage{ytableau}

% theorem styles (I like everything to have the same counter)
\theoremstyle{plain}
\newtheorem{theorem}{Theorem}[section]
\newtheorem{lemma}[theorem]{Lemma}
\newtheorem{proposition}[theorem]{Proposition}
\newtheorem{corollary}[theorem]{Corollary}
\newtheorem{conjecture}[theorem]{Conjecture}
\newtheorem{exercise}[theorem]{Exercise}
\theoremstyle{definition}
\newtheorem{definition}[theorem]{Definition}
\newtheorem{example}[theorem]{Example}
\theoremstyle{remark}
\newtheorem{remark}[theorem]{Remark}

% some numbering settings
\numberwithin{equation}{section}

\usepackage{fancyhdr}
\usepackage{lastpage}
\setlength{\headheight}{15.2pt}
\renewcommand{\footrulewidth}{0.4pt}% default is 0pt
\setlength{\footskip}{30pt}
\pagestyle{fancy}

\makeatletter
\let\ps@plain\ps@fancy 
\makeatother

\lhead{MATH 742}
\chead{$S_n$ and $\operatorname{GL}_n$}
\rhead{Spring 2023}
\lfoot{Last Revised: \today}
\cfoot{}
\rfoot{\thepage\ of \pageref{LastPage} }

\begin{document}

%%%%%%%%%%%%%%%%%%
%%%%%%%%%%%%%%%%%%
\title{Symmetric Groups and General Linear Groups}
\author{Alexander Duncan}

\maketitle

Much of this material is drawn from
\cite[\S{4,6,A}]{FultonHarris},
\cite[\S{5.12--5.19}]{Etingof},
\cite[\S{I}]{Macdonald}, and
\cite[\S{7}]{Stanley2}.

%%%%%%%%%%%%%%%%%%
%%%%%%%%%%%%%%%%%%

\section{Partitions}

Recall that a \emph{partition} $\lambda$ is a sequence of decreasing
non-negative integers
$\lambda_1 \ge \lambda_2 \ge \cdots$ that is eventually $0$.
Some standard terminology:
\begin{itemize}
\item The non-zero $\lambda_i$ are the \emph{parts} of $\lambda$.
\item The number $\ell(\lambda)$ of parts is the \emph{length} of $\lambda$.
\item The sum $|\lambda|=\sum_{i \ge 0} \lambda_i$ is the \emph{weight} of $\lambda$.
\item A ``partition of $n$'' is a partition of weight $n$.
\item $\lambda \vdash n$ means $\lambda$ is a partition of $n$.
\item The number $m_i(\lambda)$ of parts equal to $i$ is
the \emph{multiplicity} of $i$ in $\lambda$.
\item Partitions can be written $\lambda_1 + \lambda_2 + \cdots + \lambda_r$.
\item We may use the shorthand $\lambda = (1^{m_1} 2^{m_2} \cdots r^{m_r})$.
\end{itemize}

Partitions are often drawn as \emph{Young tableaux}.  This is just a
series of empty boxes where each row contains $\lambda_i$ boxes.
There are some differing conventions between different areas of math, so
read any source carefully.

Given a partition $\lambda$, the \emph{conjugate partition}
is the partition $\lambda'$ obtained by reflecting the Young diagram in
the downwards-right diagonal line.  More explicitly,
$\lambda'_i$ is the number of parts such that $\lambda_j \ge i$. 

\begin{example}
The partitions of $4$ are as follows:
\begin{center}
\begin{tabular}{ccccc}
4 & 3+1 & 2+2 & 2+1+1 & 1+1+1+1 \\
\ydiagram{4} & \ydiagram{3,1} & \ydiagram{2,2} & \ydiagram{2,1,1} &
\ydiagram{1,1,1,1}
\end{tabular}
\end{center}
\end{example}

\begin{example}
Let $\lambda = (4,4,2,2,2,1)$.
We may also write $\lambda$ as $4+4+2+2+2+1$ or $1^12^34^2$.
We have parts $\lambda_1=4$, $\lambda_2=4$,
$\lambda_3=2$,
$\lambda_4=2$,
$\lambda_5=2$, and
$\lambda_6=1$.
We have length $\ell(\lambda)=6$, weight $|\lambda|=15$,
and multiplicities $m_1(\lambda)=1$, $m_2(\lambda)=3$,
$m_3(\lambda)=0$, and $m_4(\lambda)=2$.
The Young tableau is
\begin{center}
\ydiagram{4,4,2,2,2,1}.
\end{center}
The conjugate partition $\lambda'$ is $(6,5,2,2)$.
\end{example}

We use the notation $\operatorname{Par}(n)$ to denote the set of
partitions of $n$.  And the notation $\operatorname{Par}$ for the set of
all partitions of all weights.

Recall that every permutation $\sigma \in S_n$ can be written as a
product of disjoint cycles, which is unique up to reordering of
the cycles.  Counting the cycles of length $1$, we see that the orders
of the constituent cycles give a partition $\lambda \vdash n$.
For example $(1\ 4\ 3)(2\ 8)(6\ 7) \in S_9$ has cycle type
$3+2+2+1+1$.
The following is standard in most group theory texts:

\begin{proposition}
Two permutations in $S_n$ are conjugate if and only if they have the
same cycle type.
In particular, the conjugacy classes of $S_n$ are in canonical bijective
correspondence with $\operatorname{Par}(n)$.
\end{proposition}

\section{Symmetric Functions}

There is a natural left action of the symmetric group $S_n$ on the ring of
polynomials $R=\mathbb{Z}[x_1,\ldots,x_n]$ by permuting the variables.
More precisely, there is a unique ring automorphism of $R$ defined
by $x_i \mapsto x_{\sigma{i}}$ for each variable $x_i$ and each
permutation $\sigma \in S_n$.
Alternatively, if $\sigma \in S_n$, the we have
\[
(\sigma \cdot f)(a_1,\ldots,a_n) =
f\left(a_{\sigma(1)}, \ldots, a_{\sigma}(n)\right)
\]
for $f \in R$ and $a_1,\ldots,a_n \in \mathbb{Z}$.

\begin{definition}
A polynomial $f \in \mathbb{Z}[x_1,\ldots, x_n]$ is \emph{symmetric}
if $\sigma \cdot f = f$ for all $\sigma \in \Sigma$.
We denote by $\mathsf{SF}_n$ set of symmetric polynomials
$\mathbb{Z}[x_1,\ldots,x_n]^{S_n}$ in $n$ variables.
\end{definition}

The set $\mathsf{SF}_n$ forms a graded ring
\[
\mathsf{SF}_n = \bigoplus_{d \ge 0} \mathsf{SF}_n^d
\]
where each $\mathsf{SF}_n^d = \mathbb{Z}[x_1,\ldots,x_n]_{(d)}^{S_n}$
is the subgroup of homogeneous symmetric polynomials of degree $d$.

Given a tuple $\alpha = (\alpha_1,\ldots,\alpha_n) \in \mathbb{N}^n$
we use the shorthand
\[
x^\alpha = x_1^{\alpha_1} \cdots x_n^{\alpha_n}
\]
to denote monomials in $\mathbb{Z}[x_1,\ldots,x_n]$.

Given a partition $\lambda \vdash d$ with length $\ell(\lambda) \le n$,
the \emph{monomial symmetric polynomial associated to $\lambda$}
is given by
\[
m_\lambda := \sum_{\alpha} x^\alpha
\]
where $(\alpha_1,\ldots,\alpha_n)$ ranges over all
\emph{distinct} permutations of $(\lambda_1,\ldots,\lambda_n)$.

\begin{example}
If $n=3$, then we have
\begin{align*}
m_\emptyset &= 1\\
m_1 &= x_1 + x_2 + x_3\\
m_2 &= x_1^2 + x_2^2 + x_3^2\\
m_{11} &= x_1x_2 + x_1x_3 + x_2x_3\\
m_{111} &= x_1x_2x_3 \\
m_{14} &= x_1x_2^4 + x_1^4x_2 + x_1x_3^4 + x_1^4x_3 + x_2x_3^4 +
x_2^4x_3.
\end{align*}
\end{example}

Importantly, we have the following:

\begin{proposition}
Monomial symmetric polynomials form a basis for $\mathsf{SF}_n$.
Specifically,
\[
\mathsf{SF}_n^d = \operatorname{span}_{\mathbb{Z}}
\{ m_\lambda \mid \lambda \vdash d, \ell(\lambda) \le n \}.
\]
\end{proposition}

\begin{proof}
The partitions of length at most $n$ are a system of distinct
representatives for the $S_n$-orbits of $\mathbb{N}^n$.
The monomials in $m_\lambda$ are precisely those of these orbit of
$\lambda$.
The coefficient of every monomial
in $m_\lambda$ is either $0$ or $1$ and every possible monomial
occurs in exactly one $m_\lambda$.
If $f$ is a symmetric polynomial, then the coefficient
of a monomial $x^\alpha$ must be the same as $x^{\sigma(\alpha)}$
for every $\sigma \in S_n$.  
\end{proof}

There are several notable additional families of symmetric polynomials
which we define now.

\begin{definition}
For positive integers $d$, we define
the \emph{elementary symmetric polynomial $e_d$ of degree $d$} as
\[
e_d = m_{1^d} = \sum_{1 \le i_1 < i_2 < \cdots < i_d \le n} x_{i_1}x_{i_2}\cdots
x_{i_d},
\]
the \emph{complete homogeneous symmetric polynomial $h_d$ of degree $d$} as
\[
h_d = \sum_{\lambda \vdash d} m_\lambda = \sum_{1 \le i_1 \le i_2 \le \cdots \le i_d \le n} x_{i_1}x_{i_2}\cdots
x_{i_d},
\]
and the \emph{power sum symmetric polynomial $p_d$ of degree $d$} as
\[
p_d = m_d = \sum_{1 \le i \le n} x_i^d.
\]
By convention, $e_0=h_0=p_0=1$.
\end{definition}

\begin{remark}
Note that the convention for $d=0$ is arguably not a convention
for $e_0$ and $h_0$ if one writes the definitions in a different way.
However, there are good reasons to instead define $p_0=n$.
However, this convention is incompatible with the ring of symmetric functions
discussed in the next section. 
\end{remark}

\begin{example}
If $n=3$, then we have
\begin{align*}
e_1 = h_1 = p_1 &= x_1 + x_2 + x_3\\
e_2 &= x_1x_2 + x_1x_3 + x_2x_3\\
e_3 &= x_1x_2x_3 \\
e_4 &= 0 \\
h_2 &= x_1^2 + x_2^2 + x_3^2 + x_1x_2 + x_1x_3 + x_2x_3\\
h_3 &= x_1^3 + \cdots + x_1^2x_2 + \cdots + x_1x_2x_3\\
p_2 &= x_1^2 + x_2^2 + x_3^2\\
p_3 &= x_1^3 + x_2^3 + x_3^3
\end{align*}
\end{example}

We now see some of the connections of symmetric functions and
representation theory:

\begin{proposition}
Suppose $V$ is an $n$-dimensional vector space and
$\varphi \in \operatorname{End}(V)$ has eigenvalues
$\lambda_1,\ldots,\lambda_n$.
Then the natural action of $\varphi$ on the exterior power $\Lambda^d V$
has trace
\[
\operatorname{tr}\left( \Lambda^d \varphi \right)
= e_d(\lambda_1,\ldots,\lambda_n)
\]
and the natural action of $\varphi$ on the symmetric power
$\mathcal{S}^d V$ has trace
\[
\operatorname{tr}\left( \mathcal{S}^d \varphi \right)
= h_d(\lambda_1,\ldots,\lambda_n).
\]
\end{proposition}

\begin{proof}
It suffices to work over an algebraically closed field $k$.
We prove the statement for $\Lambda^d V$ since the argument for
$\mathcal{S}^d$ is very similar. 

First, we assume that $\varphi$ is diagonalizable with eigenvectors
$v_1,\ldots, v_n$ corresponding to eigenvalues
$\lambda_1,\ldots,\lambda_n$.
Now, an eigenbasis for $\Lambda^d V$ is given by
\[
\{ v_{i_1}\wedge v_{i_2}\wedge \cdots \wedge v_{i_d} \mid
1 \le i_1 < i_2 < \cdots < i_d \le n \} \ .
\]
The corresponding eigenvalues of $\Lambda^d \varphi$ are therefore
\[
\{ \lambda_{i_1} \lambda_{i_2} \cdots \lambda_{i_d} \mid
1 \le i_1 < i_2 < \cdots < i_d \le n \} \ .
\]
The trace of $\Lambda^d \phi$ is just the sum of the eigenvalues, which
is $e_d(\lambda_1,\ldots,\lambda_n)$ as desired.

Now we consider the case where $\varphi$ is not diagonalizable.
In this case $\varphi$ has a Jordan canonical form.  Thus,
we have a basis $v_1,\ldots, v_n$ for $V$ such that $\varphi$ is
represented by $D+N$ where $D$ is a diagonal matrix and
$N$ is a lower triangular matrix.
Observe that $N(v_i)$ is a linear combination of $v_{i+1},\ldots, v_n$.

Define a total ordering on $\mathbb{N}^n$ where
$(a_1,\ldots,a_n) < (b_1,\ldots,b_n)$ if $a_i < b_i$ for the minimal $i$
on which $a_i \ne b_i$.
We now observe that
\[
(D+N)(v_{i_1}\wedge \cdots \wedge v_{i_d})
= D(v_{i_1}\wedge \cdots \wedge v_{i_d})
+ \textrm{higher terms} 
\]
where the ``higher terms'' are scalar multiples of
$v_{j_1}\wedge \cdots \wedge v_{j_d}$
where $(j_1,\ldots,j_d) > (i_1,\ldots,i_d)$.
Thus, appropriately ordered, our basis for $\Lambda^d V$
also represents $\Lambda^d V$ as a lower-triangular matrix.
The trace only depends on the diagonal entries so the result follows
from the diagonalizable case.
\end{proof}


\bibliographystyle{alpha}
\bibliography{rep_theory}

\end{document}


