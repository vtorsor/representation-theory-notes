\documentclass[12pt]{article}

\usepackage{amssymb,amsmath,amsthm}

% theorem styles (I like everything to have the same counter)
\theoremstyle{plain}
\newtheorem{theorem}{Theorem}[section]
\newtheorem{lemma}[theorem]{Lemma}
\newtheorem{proposition}[theorem]{Proposition}
\newtheorem{corollary}[theorem]{Corollary}
\newtheorem{conjecture}[theorem]{Conjecture}
\newtheorem{exercise}[theorem]{Exercise}
\theoremstyle{definition}
\newtheorem{definition}[theorem]{Definition}
\newtheorem{example}[theorem]{Example}
\theoremstyle{remark}
\newtheorem{remark}[theorem]{Remark}

% some numbering settings
\numberwithin{equation}{section}

\usepackage{fancyhdr}
\usepackage{lastpage}
\setlength{\headheight}{15.2pt}
\renewcommand{\footrulewidth}{0.4pt}% default is 0pt
\setlength{\footskip}{30pt}
\pagestyle{fancy}

\makeatletter
\let\ps@plain\ps@fancy 
\makeatother

\lhead{MATH 742}
\chead{Introduction to Representation Theory}
\rhead{Spring 2023}
\lfoot{Last Revised: \today}
\cfoot{}
\rfoot{\thepage\ of \pageref{LastPage} }

\begin{document}

%%%%%%%%%%%%%%%%%%
%%%%%%%%%%%%%%%%%%
\title{Introduction to Representation Theory}
\author{Alexander Duncan}

\maketitle

\section{Basic Concepts}

Some of the notions discussed here can also be found in
\cite[\S{1}]{Serre}.
We will use many other sources for this material,
but most (correctly!) use the module-theoretic perspective,
which we want to defer for the moment.

\begin{definition}
Let $V$ be a vector space over a field $k$.
The \emph{general linear group of $V$}, denoted $\operatorname{GL}(V)$, is the group
of invertible linear transformations $V \to V$.
For a non-negative integer $n$, we use the notation
$\operatorname{GL}_n(k) := \operatorname{GL}(k^n)$.
\end{definition}

The group $\operatorname{GL}_n(k)$ can be canonically identified with the set of $n
\times n$ invertible matrices with entries in $k$.

\begin{definition}
Suppose $G$ is a group.  A \emph{linear representation of $G$} is a
group homomorphism $\rho : G \to \operatorname{GL}(V)$ for a vector space $V$.
The \emph{degree} of a linear representation is the dimension of the
corresponding vector space $V$.
A linear representation $\rho$ is \emph{faithful} if $\rho$ is injective
as a map of sets.
\end{definition}

We often refer to the ``representation $V$'', indicating its underlying
vector space, rather than the homomorphism $\rho$.
We will sometimes write $\rho_g$ instead of $\rho(g)$ as it is easier
to read.  Later, we will simply write ``$g$'' for $\rho(g)$ when there
is no danger of confusion.

\begin{example}
For any group $G$,
the \emph{zero representation} is the representation of degree
$0$ (it is unique up to isomorphism).
\end{example}

\begin{example}
For any group $G$,
the \emph{unit} or \emph{trivial representation} is the representation
\[ \rho: G \to \operatorname{GL}_1(k) \]
given by
\[
\rho(g)=\operatorname{id}_k
\]
for all $g \in G$.
\end{example}

\begin{example}
Let $G = \langle g \rangle$ be the cyclic group of order $n$.
There is a representation
\[ \rho : G \to \operatorname{GL}_1(\mathbb{C}) \]
defined by
\[
\rho(g^k)=e^{\frac{2\pi i}{n}k} \ .
\]
\end{example}

\begin{example}
Let $D_8 = \langle s,r \mid s^2, r^4, (sr)^2 \rangle$
be the dihedral group of order $8$.
There is a representation
\[ \rho : G \to \operatorname{GL}_2(k) \]
defined on generators by
\[
\rho(s)=\begin{pmatrix} 0&1\\1&0 \end{pmatrix} ,\quad
\rho(r)=\begin{pmatrix} 0&-1\\1&0 \end{pmatrix}
\]
and extended to other elements by using the fact that $\rho$
must be a group homomorphism.
Note that $\rho$ is a \emph{faithful} representation if and only if
$\operatorname{char}(k) \ne 2$.
\end{example}


\begin{example}
For any group $G$ with a left action on a set $X$,
let $V$ be the vector space with basis indexed by $X$:
\[
V = \mathrm{span} \{ e_x : x \in X \} \ .
\]
Define the \emph{permutation representation} corresponding to $X$
as the representation
\[
\rho : G \to \operatorname{GL}(V)
\]
defined such that
\[
\rho(g)(e_x) = e_{g(x)}
\]
for all $g \in G$ and $x \in X$.
\end{example}

\begin{example}
For any group $G$, the \emph{regular representation of $G$}
is the permutation representation of the action of $G$ on itself by left
multiplication.
In other words, 
$V$ is the vector space with basis indexed by $G$:
\[
V = \mathrm{span} \{ e_g : g \in G \}
\]
and the representation
\[
\rho : G \to \operatorname{GL}(V)
\]
is defined such that
\[
\rho(g)(e_h) = e_{gh}
\]
for all $g,h \in G$.
\end{example}

\begin{example} \label{ex:lin_alg_is_Z_rep_theory}
Linear representations $\rho : \mathbb{Z} \to \operatorname{GL}_n(k)$
are completely determined by the image of $\rho(1)$
since $\rho(n)=\rho(1)^n$ for all $n \in \mathbb{Z}$.
This gives a canonical bijection of linear representations of
$\mathbb{Z}$ of degree $n$ and
$n \times n$ invertible matrices.
\end{example}

\subsection{Equivariant Maps}

\begin{definition}
Let $V$ and $W$ be vector spaces over the same field.
Suppose
\begin{align*}
\rho &: G \to \operatorname{GL}(V), \textrm{ and}\\
\rho' &: G \to \operatorname{GL}(W)
\end{align*}
are linear representations of the same group $G$.
A linear map $\tau : V \to W$ is \emph{$G$-equivariant} if
\[
\rho'(g) \circ \tau = \tau \circ \rho(g)
\]
for all $g \in G$.
The two representations are \emph{similar} or \emph{isomorphic} if there
exists an invertible $G$-equivariant linear map $\tau : V \to W$.
Let $\operatorname{Hom}_k^G(V,W)$ be the subset of
$G$-equivariant linear maps $f : V \to W$.
\end{definition}

An archaic term for $G$-equivariant map is ``intertwining operator,''
but this is disappearing (thankfully in my view).

\begin{example}
If $A$ and $B$ are similar invertible matrices, then there exists
an invertible matrix $P$ such that $A=PBP^{-1}$.
Recall that $A^n=PB^nP^{-1}$ for all integers $n$.
In view of Example~\ref{ex:lin_alg_is_Z_rep_theory}, we conclude that
equivalence classes of linear representations of
$\mathbb{Z}$ of degree $n$ are in canonical bijection with
similarity classes of $n \times n$ invertible matrices.
\end{example}

\begin{example}
Let $\rho : \mathbb{Z}/4\mathbb{Z} \to
\operatorname{GL}_1(\mathbb{C})$
be the representation where $\rho(1)=(i)$
and let $\sigma : \mathbb{Z}/4\mathbb{Z} \to
\operatorname{GL}_2(\mathbb{C})$
be the representation where
\[
\sigma(1) = \begin{pmatrix} 0&-1\\1&0 \end{pmatrix} .
\]
Define $\rho'(g)=\rho(g^{-1})$
and $\sigma'(g)=\sigma(g^{-1})$ for all $g \in G$.
By considering eigenvalues, we conclude that
$\sigma'$ and $\sigma$ are equivalent, but $\rho'$ and $\rho$ are not.
\end{example}

\begin{example} \label{ex:S3_decomposition}
Let $\rho$ be the $3$-dimensional permutation representation associated to
the action of $S_3$ on $\{1,2,3\}$.
In the usual basis, we have
\[
\rho(23) = \begin{pmatrix} 1&0&0\\ 0&0&1 \\ 0&1&0 \end{pmatrix}
\textrm{ and }
\rho(123) = \begin{pmatrix} 0&0&1\\1&0&0\\0&1&0 \end{pmatrix}.
\]
Suppose $\zeta \in k$ is a primitive cube root of unity.
Via the change of basis matrix
\[
P = \begin{pmatrix} 1&1&1\\ 1&\zeta^2&\zeta\\1&\zeta&\zeta^2 \end{pmatrix} 
\]
we have a new representation $\sigma$ where
\[
\sigma(23) = \begin{pmatrix} 1&0&0\\ 0&0&1 \\ 0&1&0 \end{pmatrix}
\textrm{ and }
\sigma(123) = \begin{pmatrix} 1&0&0\\ 0&\zeta&0 \\ 0&0&\zeta^2
\end{pmatrix} .  
\]
Observe that $\rho$ and $\sigma$ are similar by construction.
\end{example}


\subsection{Subrepresentations}

\begin{definition}
Let $\rho : G \to \operatorname{GL}(V)$ be a linear representation.
A subspace $W \subset V$ is \emph{stable under the action of $G$} if
$\rho_g(w) \in W$ for all $g \in G$, $w \in W$.
Define $\rho^W : G \to \operatorname{GL}(W)$ by $\rho^W(g) := \rho(g)|_W$ for all $g
\in G$.
We say $\rho^W$ is a \emph{subrepresentation} of $\rho$.
\end{definition}

We will also say simply ``$W$ is a subrepresentation of $V$'' in the above
situation.

If $W$ is a subrepresentation of $V$ then
\[
\rho_g(v+w) = \rho_g(v) + \rho_g(w) \in \rho_g(v)+W
\]
for any $v \in V$, $w \in W$, and $g \in G$.
This gives rise to a well-defined \emph{quotient representation}
$\rho' : G \to \operatorname{GL}(V/W)$
on the quotient space $V/W$.

\begin{definition}
Given representations $(V,\rho)$, $(W,\sigma)$ of a group $G$,
the \emph{direct sum representation} $\rho \oplus \sigma$
is the representation with underlying space $V \oplus W$ given by
\[ (\rho \oplus \sigma)_g (v,w) := ( \rho_g(v), \sigma_g(v) ) \]
for all $v \in V$, $w \in W$, and $g \in G$.
\end{definition}

\begin{proposition}
Let $V$ and $W$ be representations of $G$, and let
$f : V \to W$ be a $G$-equivariant map.
Then $\ker(f)$ is a subrepresentation of $V$.
\end{proposition}

\begin{example}
In the notation of
Example~\ref{ex:S3_decomposition}, we see that $V$ can be written
as a direct sum $U \oplus W$ where
\[
U = \operatorname{span}\left\{ \begin{pmatrix} 1,1,1 \end{pmatrix}^T
\right\}
\]
is a trivial representation, and
\[
W = \operatorname{span}\left\{
\begin{pmatrix}1,\zeta,\zeta^2\end{pmatrix}^T,
\begin{pmatrix}1,\zeta^2,\zeta\end{pmatrix}^T
\right\}
\]
is a $2$-dimensional faithful representation of $S_3$.
\end{example}

\begin{example}
Suppose $G=\langle g \rangle$ is a finite cyclic group.
Recall that every matrix of finite order can be diagonalized
over $\mathbb{C}$.
Thus, every finite-dimensional complex representation of
$G$ is a direct sum of $1$-dimensional representations.
\end{example}

\begin{example}
If $V$ is a representation of $G$, define the \emph{space of invariants}
\[
V^G = \{ v \in V \mid gv=v \textrm{ for all } g \in G \}.
\]
Observe that $V^G$ is a subrepresentation of $V$ and it is a direct sum
of trivial representations.
\end{example}

\begin{definition}
A linear representation $V$ is \emph{irreducible} if $V \ne 0$ and the only
subrepresentations of $V$ are $0$ and $V$ itself.
Otherwise, the representation is \emph{reducible}.
\end{definition}

\begin{definition}
A linear representation $V$ is \emph{indecomposable} if $V \ne 0$ and
$V$ cannot be written as a direct sum $V = W_1 \oplus W_2$ where
$W_1, W_2$ are non-zero subrepresentations.
Otherwise, the representation is \emph{decomposable}.
\end{definition}

A decomposable representation is clearly reducible, but the converse may
not hold as the next example demonstrates.

\begin{example}
Let $G=\mathbb{Z}/p\mathbb{Z}$ for a prime $p$
and consider the two-dimensional representation $\rho$ over $\mathbb{F}_p$
given by
\[
\rho(1) = \begin{pmatrix} 1&1\\0&1 \end{pmatrix} .
\]
A vector $\begin{pmatrix} a\\b \end{pmatrix} $ spans a
$G$-stable subspace only if $b=0$.
The subspace spanned by $\begin{pmatrix} 1\\0 \end{pmatrix}$
is a proper non-zero subrepresentation
of $\rho$, so $\rho$ is reducible.
However, this is the only subrepresentation so $\rho$
cannot be written as a direct sum of proper subrepresentations.
We conclude that $\rho$ is indecomposable.
\end{example}

\begin{theorem}[Krull-Schmidt]
Suppose $V$ is a finite-dimensional representation of a finite group $G$.
All decompositions
\[
V \cong W_1 \oplus W_2 \oplus \cdots \oplus W_r
\]
into indecomposable representations are unique up to isomorphism
and reordering.
\end{theorem}

\begin{proof}
Deferred for now.  We will prove this in more generality when we discuss
modules.
\end{proof}

\begin{remark}
The Krull-Schmidt theorem fails for representations $G \to
\operatorname{GL}(\mathbb{Z})$, so the fact we're working
over a field is vital here.
Unfortunately, counterexamples are rather subtle!
\end{remark}

\begin{theorem}[Schur's Lemma]
If $V$ and $W$ are finite-dimensional irreducible representations
and $k$ is algebraically closed, then
\[
\operatorname{Hom}_k^G(V,W) =
\begin{cases}
0 & \textrm{ if } V \not\cong W\\
k & \textrm{ if } V \cong W.
\end{cases}
\]
\end{theorem}

\begin{proof}
Let $\phi : V \to W$ be a $G$-equivariant map.
Since $V$ is irreducible, either $\ker(\phi)=0$ or $\ker(\phi)=V$.
Since $W$ is irreducible, either $\operatorname{im}(\phi)=0$
or $\operatorname{im}(\phi)=W$.
Thus, $\phi$ is either $0$ or an isomorphism.

Thus, we are reduced to the case where $W=V$.
Now $\operatorname{End}^G_k(V)=\operatorname{Hom}_k^G(V,W)$
is a $k$-algebra under composition.
If $\phi \in \operatorname{End}^G_k(V)$ is non-zero, it must be an isomorphism.
Thus $\phi^{-1}$ is also in $\operatorname{End}^G_k(V)$.
We conclude that $\operatorname{End}^G_k(V)$ is a division algebra.

It remains to prove that every finite-dimensional division algebra $D$
over an algebraically closed field $k$ is just the field itself.
Suppose $a \in D \setminus \{0\} $,
and let $m_a(t) \in k[t]$ be its minimal polynomial.
If $m_a$ does not have degree $1$, then $m_a(t)=f(t)g(t)$
for non-constant polynomials $f,g \in k[t]$ since $k$ is algebraically
closed.
The equation $0=m_a(a)=f(a)g(a)$ forces either $f(a)=0$ or $g(a)=0$
since $D$ is a division algebra.  But this contradicts minimality of
minimal polynomial.  We conclude that $m_a(t)=t-\lambda$ for some
$\lambda \in k$.  Thus $a$ is scalar multiplication of $\lambda$
and thus an element of $k$.
\end{proof}

\begin{definition}
A linear representation $V$ is \emph{completely reducible}
if $V$ is a direct sum of irreducible representations.
\end{definition}

Suppose $V$ is completely reducible.
The Krull-Schmidt theorem tells us that there exist irreducible
representations $W_1, \ldots, W_r$, which are pairwise non-isomorphic,
such that
\begin{equation} \label{eq:isotypic_decomposition}
V = \bigoplus_{i=1}^r V_i
\end{equation}
where each $V_i$ is isomorphic to
\[
V_i = W_i^{\oplus m_i}
\]
for some positive integer $m_i$.
The expression \eqref{eq:isotypic_decomposition} is called the
\emph{isotypic decomposition of $V$}.
The subrepresentations $V_i$ are called the \emph{isotypic components of
$V$ associated to $W_i$} while the integers $m_i$ are called the
\emph{multiplicities of $W_i$ in $V$}.

Each $V_i$ is a \emph{canonical} subrepresentation of $V$,
while classifying subrepresentations isomorphic to $W_i$ may depend on
arbitrary choices.
This generalizes how the eigenspaces of a linear transformation
are canonical, while a basis of eigenvectors
may depend on arbitrary choices.

\begin{exercise}
Prove the Krull-Schmidt theorem for the case of completely reducible
representations $V$ when the field $k$ is algebraically closed.
(Hint: use Schur's Lemma.)
\end{exercise}

The following theorem is the main reason why representation theory of finite
groups in ``good'' characteristic is a completely different story from
``bad'' characteristic.
By convention, every integer is ``coprime to the characteristic of $k$''
when the characteristic is $0$.

\begin{theorem}[Maschke]
Let $G$ be a finite group of order coprime to the characteristic of $k$.
Every finite-dimensional representation of $G$ is completely reducible.
\end{theorem}

This theorem follows by induction and the following lemma:

\begin{lemma}
Let $G$ be a finite group of order coprime to the characteristic of $k$.
If $V$ is a representation of $G$ and $W$ is a subrepresentation of $V$,
then there exists a subrepresentation $U$ of $V$
such that $V= W \oplus U$.
\end{lemma}

\begin{proof}
Let $Q: V \to W$ be a projection from $V$ onto the subspace $W$.
The projection $Q$ exists for linear algebraic reasons
and is \emph{not} necessarily $G$-equivariant.
However, we will tweak it so that it \emph{is} $G$-equivariant.
Define $P : V \to W$ via
\[
P(v) = \frac{1}{|G|} \sum_{g \in G} g\left( Q\left( g^{-1} v\right)\right)
\]
for all $v \in V$.  Note that the condition on the characteristic of the
field is needed to divide by the order of the group.

We claim that $P : V \to W$ is a $G$-equivariant projection onto $W$.
Since $Q$ restricts to the identity on $W$ and $W$ is $G$-stable,
we see that
$g \circ Q \circ g^{-1}$ is a projection onto $W$
for all $g \in G$.  Thus, we have
\begin{align*}
P^2(v) &=
\frac{1}{|G|^2} \sum_{g,h \in G} gQg^{-1} hQh^{-1}v\\
&=\frac{1}{|G|^2} \sum_{g,h \in G} hQh^{-1}v\\
&=\frac{1}{|G|} \sum_{h \in G} hQh^{-1}v = P(v)
\end{align*}
and conclude that $P$ is a projection.  Since
$P|_W=\operatorname{id}_W$, we conclude it is a projection onto $W$.

Now we show that $P$ is equivariant.
This follows from a reindexing trick $j=h^{-1}g$ in the following:
\begin{align*}
P(hv) &=
\frac{1}{|G|} \sum_{g \in G} gQg^{-1} (hv)\\
&=\frac{1}{|G|^2} \sum_{j \in G} hjQj^{-1}v\\
&=h\left(\frac{1}{|G|} \sum_{j \in G} jQj^{-1}v\right) = hP(v).
\end{align*}
The kernel of $P$ is the desired $G$-stable complement $U$.
\end{proof}

\subsection{Constructions}

Suppose $(V,\rho)$ and $(W,\sigma)$ are finite-dimensional linear
representations of a group $G$ over a field $k$.

Let $\operatorname{Hom}_k(V,W)$ denote the vector space of linear transformations
$\tau : V \to W$.
Let $V^\vee$ be the vector space dual to $V$.
Let $V \otimes W$ be the tensor product over $k$.

\begin{proposition}
The vector space $\operatorname{Hom}_k(V,W)$ has a canonical structure of a linear
representation $\tau$ where
\[
 \tau_g(f) = \sigma(g) \circ f \circ \rho(g)^{-1}
\]
for all $f : V \to W$ and $g \in G$.
\end{proposition}

The $G$-action defined above has the very useful property that
the set of $G$-equivariant homomorphisms is precisely the set of invariants of the
$G$-action on the set of homomorphisms.  In other words,
\[
\operatorname{Hom}^G_k(V,W) = \operatorname{Hom}_k(V,W)^G .
\]

\begin{proposition}
The dual space $V^\vee$ has a canonical structure of a linear
representation $\rho^\vee$ where
\[
 (\rho^\vee)_g(f) = f \circ \rho(g)^{-1}
\]
for $f : V \to k$ and $g \in G$.
\end{proposition}

Note that the above proposition is a special case of the action on
homomorphisms when $W=k$ has a trivial $G$-action.

\begin{proposition}
The tensor product $V \otimes W$ has a canonical structure of a linear
representation $\rho \otimes \sigma$ where
\[
(\rho \otimes \sigma)_g(v \otimes w) = \rho_g(v) \otimes \sigma_g(w)
\]
for $v \in V$, $w \in W$ and $g \in G$.
\end{proposition}

Note that the above proposition agrees with the $G$-action one
obtains from the canonical isomorphism
\[
\operatorname{Hom}_k(W,V) \cong W^\vee \otimes V. 
\]

\begin{exercise}
For a vector space $V$,
determine the actions of $G$ on
$\mathcal{T}^d(V)$, $\mathcal{S}^d(V)$, $\Lambda^d(V)$,
$\operatorname{Sym}^n(V)$, and $\operatorname{Alt}^n(V)$.
Verify that the various canonical maps are equivariant.
\end{exercise}

%%%%%%%%%%%%%%%%%%
%%%%%%%%%%%%%%%%%%

\bibliographystyle{alpha}
\bibliography{rep_theory}

\end{document}


